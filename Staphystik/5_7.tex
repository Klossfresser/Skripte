\section{Thermodynamik}
\subsection{Zustände}
Im folgenden bedeutet \uline{Zustand} in der Regel: \uline{Gleichgewichts-Makrozustand},
\begin{itemize}[align=left,labelwidth=\widthof{--},labelsep=1ex,labelindent=0pt,leftmargin=\labelsep+\labelindent+\labelwidth]
  \item[--] zeitlich unveränderlich, wenn äußere Parameter nicht verändert werden,
  \item[--] gekennzeichnet durch einige makroskopische Angaben (Energie, Volumen, \dots)
\end{itemize}

\begin{itemize}[align=left,labelwidth=\widthof{\uline{Zustandsvariable}:},labelsep=1ex,labelindent=0pt,leftmargin=\labelsep+\labelindent+\labelwidth]
  \item[\uline{Zustandsgrößen}:] Größen, die in einem Zustand eindeutig festgelegt sind , zum Beispiel Energie, Entropie, Volumen, \dots; nicht: Arbeit, Wärme. Es gibt intensive/ extensive Zustandsgrößen.
  \item[\uline{Zustandsvariable}:] Ein Satz von Zustandsgrößen, der ausreicht, um alle anderen Zustandsgrößen eindeutig festzulegen. \uwave{Beispiel}: Ideales Gas mit Variablen $E$, $V$, $N$; oder alternativ $T$, $V$, $N$; und so weiter.
\end{itemize}

\subsection{Arbeit und Wärme}
Unterteile Zustandsgrößen in \uline{äußere Parameter} (\uline{Kontrollparameter}) (können vom Experimentator will\-kür\-lich von außen eingestellt werden, zum Beispiel Volumen, gegebenenfalls Teilchenzahl) und andere, \uline{innere Größen}, die auch von den inneren Eigenschaften abhängt. Oft genügt neben den äußeren Parametern eine weitere Variable (zum Beispiel Temperatur), um den Zustand festzulegen (\glqq einfaches System\grqq).\\
Energie kann auf zwei Arten geändert werden:
\begin{itemize}[align=left, labelwidth=\widthof{-- \uline{Wärme}:}, labelsep=1ex, labelindent=0pt, leftmargin=\labelsep+\labelindent+\labelwidth]
  \item[-- \uline{Arbeit}:] Veränderung der Kontrollparameter
  \item[-- \uline{Wärme}:] Energieübertragung ohne Änderung der Kontrollparameter
\end{itemize}
Arbeit: $\delta W = f\cdot \dd{X}$ $f$ intensiv, $X$ extensiv (für kontrollierte, reversible Arbeit)\\
\uwave{Beispiele}:
\begin{itemize}[align=left, labelwidth=\widthof{4)}, labelsep=1ex, labelindent=0pt, leftmargin=\labelsep+\labelindent+\labelwidth]
  \item[1)]  Gas: Expansionsarbeit  $\delta W = - F \dd{X} = -\frac{F}{A} A \dd{x} = - p \dd{V}$ mit Druck \fbox{$p=\frac{F}{A}$}, Kraft  des Systems auf Wand. $\delta W > 0$, wenn $\dd{V} > 0$ 
\begin{tikzpicture}[baseline=0.3cm]
  \fill[fill=white!90!black] (0,0) rectangle (5,1);
  \draw[thick] (7,0) to (0,0) to (0,1) to (7,1);
  \node at (2.5,0.5) {$V$};
  \draw[dashed] (5,0.05) rectangle (6,0.95);
  \draw[thick] (5,0.05)  to (5,0.95);
  \draw[decorate,decoration={brace,amplitude=3pt},yshift=2pt] (5,0) to (6,0);
  \node[anchor=south] at (5.5,0.1) {$\dd{X}$};
  \draw[->] (5,0.7) to (5.5,0.7);
\end{tikzpicture}

  \item[2)] Gummiband: Dehnungsarbeit $\delta W = F\cdot\dd{l}$, $\dd{l}$: Längenänderung, $f>0$: Von außen angewandte Kraft Hier: $\delta W >0$, wenn $\dd{l}>0$
  \item[3)] Magnetisches System: $\delta W = \vec{B}\cdot\dd{\norBra{\vec{M}V}}$ $\vec{B}$: magnetische Flussdichte, $\vec{M}V$: Magnetisierung (hier magnetisches Moment pro Volumen)
  \item[4)] Chemische Arbeit: $\delta W = \mu \dd{N}$
\end{itemize}
\uline{1. Hauptsatz der Thermodynamik}:\\
\fbox{\parbox{0.98\textwidth}{Es gibt in jedem System eine extensive Zustandsgröße \glqq innere Energie\grqq\ ($E$ oder $U$), die beim abgeschlossenen System erhalten bleibt und sich gemäß \fbox{$\dd{E} = \delta W + \delta{Q}$} verhält.}}\\
(Kann als Energieerhaltungssatz beziehungsweise Definition von Wärme aufgefasst werden. Schreibweise \glqq$\dd{E}$\grqq\ deutet an, dass $E$ eine Zustandsgröße ist und daher $\dd{E}$ ein exaktes Differential ist.)\\
\uline{Mikroskopische Interpretation}\\
Energie $E = \angBra{E} = \sum\limits_j P_j E_j$
\begin{tabbing}
\hspace{4em} \= \kill
$\rightarrow$\> $\dd{E} = \sum\limits_j P_j \dd{E_j} + \sum\limits_j E_j \dd{P_j}$\\
Arbeit: \> $\sum\limits_j P_j \dd{E_j}$\\
Wärme: \> $\sum\limits_jE_j \dd{P_j}$
\end{tabbing}
$\sum\limits_j P_j \dd{E_j} = \sum\limits_j P_j \ppv{E_j}{V} \dd{V} = - p \dd{V}$ mit $-p \mDef \sum\limits_j P_j \ppv{E_j}{V}$\\
(wenn $E_j$ nur von $V$ abhängig; gilt zum Beispiel nicht für Arbeit mit Reibungsverlusten, zum Beispiel Rühren)
\begin{itemize}[align=left, labelwidth=6em, labelsep=1ex, labelindent=0pt, leftmargin=\labelsep+\labelindent+\labelwidth, listparindent=\leftmargin]
\item[$\rightarrow$ Arbeit] $\eDef$ Änderung der Energieniveaus $E_j$
\item[$\rightarrow$ Wärme] $\eDef$ Änderung der Besetzungswahrscheinlichkeiten $P_j$
\end{itemize}
Achtung: Die geänderten $P_j$, $E_j$ entsprechen hier nicht unbedingt einem Gleichgewichtszustand. Ein solcher muss sich erst einstellen.

\subsection{Thermodynamische Prozesse}
Prozess $=$ Änderung von Zustandsgrößen

\begin{itemize}[align=left, labelwidth=\widthof{\uline{Quasistatischer} Prozess:}, labelsep=1ex, labelindent=0pt, leftmargin=\labelsep+\labelindent+\labelwidth, listparindent=\leftmargin]
\item[\uline{Quasistatischer} Prozess:] langsam genug, so dass System immer im Gleichgewicht ist; das heißt langsamer als die Relaxationszeit des Systems\\
$\rightarrow$ Abfolge von Gleichgewichtszuständen
\item[\uline{Reversibler} Prozess:] umkehrbar; genauer (Plancksche Definition): es ist möglich, den Prozess rück\-gängig zu machen und dazu auch alle benutzten Hilfsmittel in den Anfangszustand zu versetzen.
\item[\uline{Adiabatischer} Prozess:] ohne Wärmeaustausch; thermisch isoliertes System
\end{itemize}

\uwave{Beispiele}:

\tikzset{every node/.style = {draw = none, fill = none}}
\setlength{\tabcolsep}{0.5cm}
\renewcommand{\arraystretch}{2}

\begin{table}[H]
\centering
\begin{tabular}{c c}
  \begin{tikzpicture}[baseline=0cm]
  \def\sysLen{3};
  \def\sysHei{1};
  \def\pinRad{0.1};
  \def\pinDis{0.2};
  \def\pinExt{0.9};
  \def\filCol{white!90!black};
  \def\arrLen{2};
  \def\picDis{0.5};
  \fill[\filCol] (0,0) rectangle (\sysLen/2,\sysHei);
  \draw[thick] (0,0) rectangle (\sysLen,\sysHei);
  \draw[thick] (\sysLen/2,-\pinDis) circle (\pinRad) to (\sysLen/2,\sysHei);
  \draw[->] (\sysLen+\picDis,\sysHei/2) to ++(\arrLen,0);
  \fill[\filCol] (\sysLen+\arrLen+2*\picDis,0) rectangle ++(\sysLen,\sysHei);
  \draw[thick] (\sysLen+\arrLen+2*\picDis,0) rectangle ++(\sysLen,\sysHei);
  \draw[thick] (3*\sysLen/2+\arrLen+2*\picDis,-\pinDis-\pinExt) circle (\pinRad) to (3*\sysLen/2+\arrLen+2*\picDis,\sysHei-\pinExt);
\end{tikzpicture}

  &
  \setlength{\tabcolsep}{1pt}
  \begin{tabular}{c}plötzliche Expansion\\ nicht quasistatisch\\ nicht reversibel\end{tabular}
  \setlength{\tabcolsep}{1cm}
  \\
  \begin{tikzpicture}[baseline=0.3cm]
  \def\sysLen{3};
  \def\sysHei{1};
  \def\pinRad{0.1};
  \def\pinDis{0.2};
  \def\pinLen{0.2};
  \def\pisDis{0.05};
  \def\pinExt{0.9};
  \def\filCol{white!90!black};
  \def\arrDis{0.2};
  \def\smaArr{0.4};
  \def\arrLen{2};
  \def\picDis{0.5};
  \fill[\filCol] (0,0) rectangle (\sysLen/2,\sysHei);
  \draw[thick] (\sysLen,0) to ++(-\sysLen,0) to ++(0,\sysHei) to ++(\sysLen,0);
  \draw[thick] (\sysLen/2,\pisDis) to ++(0,\sysHei-2*\pisDis);
  \draw[thick] (\sysLen/2,\sysHei/2) to ++(\pinLen+\pinDis,0) circle (\pinRad);
  \draw[<->] (\sysLen/2+\pinLen+\pinDis-\smaArr/2,\sysHei/2+\arrDis) to ++(\smaArr,0);
  \draw[->] (\sysLen+\picDis,\sysHei/2) to node[above] {Arbeit} node[below] {Wärme} ++(\arrLen,0);
  \fill[\filCol] (\sysLen+\arrLen+2*\picDis,0) rectangle ++(3*\sysLen/4,\sysHei);
  \draw[thick] (\sysLen+\arrLen+2*\picDis+\sysLen,0) to ++(-\sysLen,0) to ++(0,\sysHei) to ++(\sysLen,0);
  \draw[thick] (\sysLen+\arrLen+2*\picDis+3*\sysLen/4,\pisDis) to ++(0,\sysHei-2*\pisDis);
  \draw[thick] (\sysLen+\arrLen+2*\picDis+3*\sysLen/4,\sysHei/2) to ++(\pinLen+\pinDis,0) circle (\pinRad);
  \draw[<->] (\sysLen+\arrLen+2*\picDis+3*\sysLen/4+\pinLen+\pinDis-\smaArr/2,\sysHei/2+\arrDis) to ++(\smaArr,0);
\end{tikzpicture}

  &
  \setlength{\tabcolsep}{1pt}
  \begin{tabular}{c}quasistatisch\\ und reversibel\\ (bei langsamer Bewegung)\end{tabular}
  \setlength{\tabcolsep}{1cm}
  \\
  \begin{tikzpicture}[baseline=0.5cm]
  \def\sysLen{3};
  \def\sysHei{1};
  \def\pinRad{0.1};
  \def\pinDis{0.2};
  \def\pinExt{0.9};
  \def\filCol{white!90!black};
  \def\arrLen{2};
  \def\picDis{0.5};
  \fill[\filCol] (0,0) rectangle (\sysLen/2,\sysHei);
  \draw[thick] (0,0) rectangle ++(\sysLen,\sysHei);
  \foreach \x in {0,...,6} \draw[thick] (\sysLen/2+\x*\sysLen/16,-\pinDis) circle (\pinRad) to (\sysLen/2+\x*\sysLen/16,\sysHei);

  \draw[->] (\sysLen+\picDis,\sysHei/2) to ++(\arrLen,0);

  \fill[\filCol] (\sysLen+\arrLen+2*\picDis,0) rectangle ++(10*\sysLen/16,\sysHei);
  \draw[thick] (\sysLen+\arrLen+2*\picDis,0) rectangle ++(\sysLen,\sysHei);
  \foreach \x in {0,...,6} \draw[thick] (3*\sysLen/2+\arrLen+2*\picDis+\x*\sysLen/16,{-\pinDis-\pinExt*(1/(1-\pinExt*exp(-(\x+2)/3))-1)}) circle (\pinRad) to (3*\sysLen/2+\arrLen+2*\picDis+\x*\sysLen/16,{\sysHei-\pinExt*(1/(1-\pinExt*exp(-(\x+2)/3))-1)});
\end{tikzpicture}

  &
  \setlength{\tabcolsep}{1pt}
  \begin{tabular}{c}quasistatisch, \\ nicht reversibel\end{tabular}
  \setlength{\tabcolsep}{1cm}

\end{tabular}
\end{table}
Adiabatischer Prozess: muss \glqq schnell genug\grqq\ ablaufen, um Wärmeaustausch zu verhindern.\\
\uwave{Beispiele}:
\setlength{\tabcolsep}{1cm}
\renewcommand{\arraystretch}{4}
\begin{table}[H]
  \centering
  \begin{tabular}{c c}
    \begin{tikzpicture}[baseline=0.3cm]
  \def\sysLen{3};
  \def\sysHei{1};
  \def\pinRad{0.1};
  \def\pinDis{0.2};
  \def\pisDis{0.05};
  \def\pinLen{0.9};
  \def\filCol{white!90!black};
  \def\arrLen{2};
  \def\picDis{0.5};
  \def\borDis{0.1};
  \def\smaArr{0.4};
  \def\arrDis{0.2};
  \fill[\filCol] (0,0) rectangle (\sysLen/2,\sysHei);
  \draw[thick] (\sysLen,0) to ++(-\sysLen,0) to ++(0,\sysHei) to ++(\sysLen,0);
  \draw[thick] (\sysLen,-\borDis) to ++(-\sysLen-\borDis,0) to ++(0,\sysHei+2*\borDis) to ++(\sysLen+\borDis,0);
  \draw[thick] (\sysLen/2,\pisDis) to ++(0,\sysHei-2*\pisDis);
  \draw[thick] (\sysLen/2,\sysHei/2) to ++(\pinLen+\pinDis,0) circle (\pinRad);
  \draw[<->] (\sysLen/2+\pinLen+\pinDis-\smaArr/2,\sysHei/2+\arrDis) to ++(\smaArr,0);
\end{tikzpicture}

    & reversible, adiabatische Arbeit $\delta W \stackrel{<}{>} 0$\\
    \begin{tikzpicture}[baseline=0.3cm]
  \def\sysLen{3};
  \def\sysHei{1};
  \def\pinRad{0.1};
  \def\pinDis{0.2};
  \def\pisDis{0.05};
  \def\pinLen{0.9};
  \def\filCol{white!90!black};
  \def\arrLen{2};
  \def\picDis{0.5};
  \def\borDis{0.1};
  \def\smaArr{0.2};
  \def\arrDis{0.2};
  \fill[\filCol] (0,0) rectangle (\sysLen/2,\sysHei);
  \draw[thick] (\sysLen,0) to ++(-\sysLen,0) to ++(0,\sysHei) to ++(\sysLen,0);
  \draw[thick] (\sysLen,-\borDis) to ++(-\sysLen-\borDis,0) to ++(0,\sysHei+2*\borDis) to ++(\sysLen+\borDis,0);
  \draw[thick] (\sysLen/2,0) to ++(0,\sysHei);
  \draw[thick] (\sysLen/2-\pinLen,\sysHei/2+\pinRad) circle(\pinRad);
  \draw[thick] (\sysLen/2-\pinLen,\sysHei/2-\pinRad) circle(\pinRad);
  \draw[thick] (\sysLen/2-\pinLen,\sysHei/2) to ++(2*\pinLen,0);
  \draw[->] (\sysLen/2+\pinLen-\smaArr,\sysHei/2) arc(-180:170:\smaArr);
\end{tikzpicture}

    & Rührarbeit, adiabatisch und irreversibel $\delta W > 0$
  \end{tabular}
\end{table}
\setlength{\tabcolsep}{1pt}
\renewcommand{\arraystretch}{1}
Achtung: In anderen Gebieten, besonders Quantenmechanik, bedeutet \glqq adiabatisch\grqq\ einen langsamen, reversiblen Prozess, bei dem die Besetzungswahrscheinlichkeiten konstant bleiben.


\subsection{Temperatur}
\label{sec:Temperatur5}
Empirisch: Systeme in Kontakt (diathermische Wände) können Wärme austauschen.\\
\uline{Thermisches Gleichgewicht}: kein Netto-Austausch von Wärme\\
\uline{Nullter Hauptsatz}:\\
\fbox{\parbox{0.98\textwidth}{Wenn Systeme $A$, $B$ im thermischen Gleichgewicht sind, sowie $B$, $C$ ebenso, dann sind auch $A$, $C$ im thermischen Gleichgewicht.}}
\begin{itemize}[align=left, labelwidth=4em, labelsep=1ex, labelindent=0pt, leftmargin=\labelsep+\labelindent+\labelwidth, listparindent=\leftmargin]
\item[$\rightarrow$] Ermöglicht die Definition der Temperatur durch Einteilen der physikalischen Systeme in Äqui\-valenz\-klas\-sen von Systemen im thermischen Gleichgewicht ($\eDef$ gleiche Temperatur).
\item[$\rightarrow$]\uline{Empirische Temperatur}
\item[$\rightarrow$]im Einklang mit der statistischen Definition $T^{-1}\mDef \ppv{S}{E}$.
\end{itemize}


\subsection{Differentiale}
Betrachte Funktion $F: \mathds{R}^n \to \mathds{R}$\\
\uline{Totales Differential}: \fbox{$\dd{f} = \sum\limits_j \ppv{f}{x_j} \dd{x_j}$} ($x=\norBra{x_1,\dots,x_n}\in \mathds{R}^n$)\\
Anschaulich: Änderung von $f$ bei kleinen Änderungen $\dd{x_j}$.\\
Saubere Definition: $\dd{f}$ ist eine lineare Abbildung $\mathds{R}^n \to \mathds{R}$, die im Limes $\dd{x}\to 0$ die Abbildung $f\norBra{x_0+\dd{x}}-f\norBra{X_0}$ approximiert. Hier sind $x_0$, $\dd{x}$ $\in \mathds{R}^n$.\\
Kurvenintegral:\\
$\left.\begin{array}{l r}\text{Differential}&\dd{f} \\ \text{Kurve}&C: \mathds{R} \to \mathds{R}^n; \tau \mapsto x\norBra{\tau}\end{array} \right\} \to \int\limits_C \dd{f} = \int\limits_{\tau_1}^{\tau_2} \sum\limits_j \ppv{f}{x_j}\dot{x}_j \dd{\tau}$\\
mit $\dot{x}_j = \ddv{x_j}{\tau}$\\
$\int\limits_C \dd{f} = f\norBra{x\norBra{\tau_2}}-f\norBra{x\norBra{\tau_1}}$ wegunabhängig\\
\uline{Verallgemeinerung}: Ersetze $\ppv{f}{x_j}$ durch Funktionen $g_j: \mathds{R}^n \to \mathds{R}$, die nicht unbedingt partielle Ableitungen einer Funktion sind. $\rightarrow$ im Allgemeinen $\ppv{g_s}{x_k} \neq \ppv{g_k}{x_s}$.\\
$\rightarrow$ Definiere \uline{Differentialform} (\glqq 1-Form\grqq\ , \glqq Pfaffsche Form\grqq\ ):\\
\fbox{$\delta G \mDef \sum\limits_j g_j \dd{x_j}$} $\rightarrow$ Kurvenintegrale $\int\limits_C\delta G = \int\limits_{\tau_1}^{\tau_2} \sum\limits_j g_j \dot{x}_j \dd{\tau}$.\\
\glqq$\delta G$\grqq\ ist nur Schreibweise und impliziert nicht die Existenz einer Funktion $G$. Wenn jedoch \fbox{$\ppv{g_j}{x_k} = \ppv{g_k}{x_j}$} $\forall j,k$, dann ist $\delta G$ ein \uline{exaktes Differential} und wir können schreiben$\delta G = \dd{G}$, $g_j = \ppv{G}{x_j}$. (Hier wurde angenommen, dass Definitionsbereich einfach geschlossen ist.) Dann heißt $G$ \uline{Potential} zu $\delta G$. Kurvenintegrale im Allgemeinen wegabhängig.\\
\uwave{Beispiel}:\\
$\delta G = y \dd{x} + \dd{y}$, $g_x = y$, $g_y = 1$, $\ppv{g_y}{x} = 0$, $\ppv{g_x}{y} = 1$
\begin{figure}[H]
  \centering
  \begin{tikzpicture}
  \draw[->] (-0.15,0) to (10.25,0);
  \draw[->] (0,-0.15) to (0,5.25);
  \node[anchor=south] at (0,5.35) {$y$};
  \node[anchor=west] at (10.35,0) {$x$};
  \draw[thick,domain=0:5] plot (\x,\x);
  \node[anchor=south] at (2.5,2.6) {$C_1$};
  \draw[thick,domain=0:5] plot (\x,{\x^2/5});
  \node[anchor=west] at (2.5,1.1) {$C_2$};
  \fill[thick] (5,5) circle (0.1);
  \node[anchor=west] at (5.1,5) {$(1,1)$};
\end{tikzpicture}

\end{figure}

\setlength{\tabcolsep}{5pt}
\renewcommand{\arraystretch}{1}
\begin{table}[H]
  \centering
  \begin{tabular}{c c c c c}
    $C_1$:&$\tau\in [0,1]$&$x\norBra{\tau} =\tau$&$y\norBra{\tau}=\tau$&$\int\limits_{C_1}\delta G = \int\limits_0^1 \norBra{g_x \dot{x} + g_y \dot{y}}\dd{\tau} = \int\limits_0^1 \norBra{y\cdot 1+1}\dd{\tau} = \frac32$\\
    $C_2$:&$\tau\in [0,1]$&$x\norBra{\tau} = \tau$&$y\norBra{\tau} = \tau^2$&$\int\limits_{C_2}\delta G = \int\limits_0^1 \norBra{y\cdot 1 + 2\tau}\dd{\tau} = \int\limits_0^1\norBra{\tau^2+2\tau}\dd{\tau} = \frac43$
  \end{tabular}
\end{table}

\uline{Rechenregeln} ($a,b\in \mathds{R}$ konstant): $\dd{a} = 0$, $\dd{\norBra{a f + b g}} = a\dd{f} + b \dd{g}$, $\dd{\norBra{fg}} =g\dd{f} + f\dd{g}$\\
$\lambda\norBra{x}$ (überall $\neq 0$) heißt integrierender Faktor von $\delta G$, wenn $\lambda \delta G$ ein exaktes Differential ist. Dann gilt $\ppv{}{x_j}\lambda g_k = \ppv{}{x_k}\lambda g_j$.\\
Beispiel: $e^{x}\norBra{y \dd{x} + x\dd{x}} = \dd{e^x y}$\\
In $3$ Dimensionen: $\delta G = \vec{g}\cdot\vec{\dd{x}}$,
$\vec{g} = \norBra{\begin{array}{c} g_1\\g_2\\g_3\end{array}}$,
$\vec{\dd{x}} = \norBra{\begin{array}{c}\dd{x_1}\\ \dd{x_2}\\ \dd{x_3}\end{array}}$\\
Exakt, wenn $\vec{\nabla}\times \vec{g} = 0$.\\
Integrierender Faktor existiert, wenn $\vec{\nabla}\times \lambda \vec{g} = \lambda\vec{\nabla}\times \vec{g} - \vec{g}\times\vec{\nabla}\lambda = 0$ mit $\lambda \neq 0$. $\rightarrow$ $\vec{g}\cdot\lambda\vec{\nabla}\times\vec{g} = 0$ $\rightarrow$ \fbox{$\vec{g}\cdot\vec{\nabla}\times\vec{g} = 0$} (Frobenius-Bedingung)


\subsection{Entropie und 2. Hauptsatz}
Empirisch gilt in adiabatisch eingeschlossenen Systemen: Es gibt \uline{reversible} und \uline{irreversible} Prozesse.\\
$\rightarrow$ Fasse alle (Gleichgewichts-)Zustände, die durch reversible Prozesse ineinander Überführt werden können, zu einer Äquivalenzklasse zusammen und führe eine Zustandsgröße $S$ ein (empirische Entropie), die für alle Zustände der Äquivalenzklasse denselben Wert hat.\\
Ordnungsschema: $S$ so gewählt, dass $S$ durch irreversible Prozesse erhöht wird (Konvention).\\
Bei der Wegnahme von einschränkenden Bedingungen (\glqq Enthemmung\grqq) strebt das System dem Zustand höchster Entropie zu (im Einklang mit dem Prinzip der maximalen Entropie der Statistischen Physik).

\setlength{\tabcolsep}{0.5cm}
\renewcommand{\arraystretch}{3}
\begin{table}[H]
  \centering
  \begin{tabular}{c c}
    \begin{tikzpicture}[baseline=0cm]
  \def\sysLen{3};
  \def\sysHei{1};
  \def\pinRad{0.1};
  \def\pinDis{0.2};
  \def\pisDis{0.05};
  \def\pinLen{2.3};
  \def\filCol{white!90!black};
  \def\blaKkk{black};
  \def\arrLen{2};
  \def\picDis{0.5};
  \def\borDis{0.1};
  \def\smaArr{0.4};
  \def\arrDis{0.2};
  \def\befFil{0.6};
  \def\aftFil{0.3};
  \def\picLen{5};

  \fill[\filCol] (0,0) rectangle ++(\befFil*\sysLen,\sysHei);
  \draw (0,0) rectangle ++(\sysLen,\sysHei);
  \draw (-\borDis,-\borDis) rectangle ++ (\sysLen+2*\borDis,\sysHei+2*\borDis);
  \fill[\blaKkk] (\befFil*\sysLen,0) rectangle ++(\borDis,\sysHei);
  \draw (\befFil*\sysLen+\borDis,\sysHei/2) to ++(\pinLen,0) circle (\pinRad);
  \draw[<-] (\befFil*\sysLen+\borDis+\pinLen-\smaArr/2,\sysHei/2+\arrDis) to ++(\smaArr,0);
  \draw[->] (\picLen+\picDis,\sysHei/2) to ++(\arrLen,0);

  \fill[\filCol] (\picLen+2*\picDis+\arrLen,0) rectangle ++(\aftFil*\sysLen,\sysHei);
  \draw (\picLen+2*\picDis+\arrLen,0) rectangle ++(\sysLen,\sysHei);
  \draw (\picLen+2*\picDis+\arrLen-\borDis,-\borDis) rectangle ++(\sysLen+2*\borDis,\sysHei+2*\borDis);
  \fill[\blaKkk] (\picLen+2*\picDis+\arrLen+\aftFil*\sysLen,0) rectangle ++(\borDis,\sysHei);
  \draw (\picLen+2*\picDis+\arrLen+\aftFil*\sysLen+\borDis,\sysHei/2) to ++(\pinLen,0) circle (\pinRad);
  \draw[white] (2*\picLen+2*\picDis+\arrLen,0) to ++(0,\sysHei);
\end{tikzpicture}

    &
    $S' = S$\\
    \begin{tikzpicture}[baseline=0cm]
  \def\sysLen{3};
  \def\sysHei{1};
  \def\pinRad{0.1};
  \def\pinDis{0.2};
  \def\pisDis{0.05};
  \def\pinLen{2.3};
  \def\filCol{white!90!black};
  \def\blaKkk{black};
  \def\arrLen{2};
  \def\picDis{0.5};
  \def\borDis{0.1};
  \def\smaArr{0.4};
  \def\arrDis{0.2};
  \def\befFil{0.3};
  \def\aftFil{0.4};
  \def\picLen{5};

  \fill[\filCol] (0,0) rectangle ++(\befFil*\sysLen,\sysHei);
  \draw (0,0) rectangle ++(\sysLen,\sysHei);
  \draw (-\borDis,-\borDis) rectangle ++ (\sysLen+2*\borDis,\sysHei+2*\borDis);
  \fill[\blaKkk] (\befFil*\sysLen,0) rectangle ++(\borDis,\sysHei);
  \draw[->] (\befFil*\sysLen+\borDis,\arrDis) to ++(\smaArr,0);
  \draw[->] (\picLen+\picDis,\sysHei/2) to ++(\arrLen,0);

  \fill[\filCol] (\picLen+2*\picDis+\arrLen,0) rectangle ++(\sysLen,\sysHei);
  \draw (\picLen+2*\picDis+\arrLen,0) rectangle ++(\sysLen,\sysHei);
  \draw (\picLen+2*\picDis+\arrLen-\borDis,-\borDis) rectangle ++(\sysLen+2*\borDis,\sysHei+2*\borDis);
  \fill[\blaKkk] (\picLen+2*\picDis+\arrLen+\aftFil*\sysLen,\borDis) rectangle ++(\borDis,\sysHei-\borDis);
  \draw[white] (2*\picLen+2*\picDis+\arrLen,0) to ++(0,\sysHei);
\end{tikzpicture}

    &
    $S' > S$\\
    \begin{tikzpicture}[baseline=0cm]
  \def\sysLen{3};
  \def\sysHei{1};
  \def\pinRad{0.1};
  \def\pinDis{0.2};
  \def\pisDis{0.05};
  \def\pinLen{2.3};
  \def\filCol{white!90!black};
  \def\blaKkk{black};
  \def\arrLen{2};
  \def\picDis{0.5};
  \def\borDis{0.1};
  \def\smaArr{0.4};
  \def\arrDis{0.2};
  \def\befFil{0.3};
  \def\aftFil{0.4};
  \def\picLen{5};
  \def\innSep{0.3};
  \def\innLen{0.6};
  \def\leiLen{0.45};
  \def\leiAng{20};
  \def\cloAng{2};

  \draw (0,0) rectangle ++(\sysLen,\sysHei);
  \draw (-\borDis,-\borDis) rectangle ++ (\sysLen+2*\borDis,\sysHei+2*\borDis);
  \draw (\innSep,\innSep) rectangle (\innSep+\innLen,\sysHei-\innSep);
  \draw (\innSep-\borDis,\innSep-\borDis) rectangle (\innSep+\innLen+\borDis,\sysHei-\innSep+\borDis);
  \draw (\sysLen-\innSep,\sysHei-\innSep) rectangle (\sysLen-\innSep-\innLen,\innSep);
  \draw (\sysLen-\innSep+\borDis,\sysHei-\innSep+\borDis) rectangle (\sysLen-\innSep-\innLen-\borDis,\innSep-\borDis);
  \foreach \x in {1,2,3} \draw (\innSep+\innLen,{(\sysHei-2*\innSep)/4*\x+\innSep}) to ++(\leiLen,0) to ++(\leiAng:{(\sysLen-2*\innSep-2*\innLen-2*\leiLen)}) (\sysLen-\innSep-\innLen,{(\sysHei-2*\innSep)/4*\x+\innSep}) to ++(-\leiLen,0);


  \draw[->] (\picLen+\picDis,\sysHei/2) to node[above] {Enthemmung} ++(\arrLen,0);
  \tikzset{shift={(\picLen+2*\picDis+\arrLen,0)}}

  \draw (0,0) rectangle ++(\sysLen,\sysHei);
  \draw (-\borDis,-\borDis) rectangle ++ (\sysLen+2*\borDis,\sysHei+2*\borDis);
  \draw (\innSep,\innSep) rectangle (\innSep+\innLen,\sysHei-\innSep);
  \draw (\sysLen-\innSep,\sysHei-\innSep) rectangle (\sysLen-\innSep-\innLen,\innSep);

  \foreach \x in {1,2,3} \draw (\innSep+\innLen,{(\sysHei-2*\innSep)/4*\x+\innSep}) to ++(\leiLen,0) to ++(\cloAng:{(\sysLen-2*\innSep-2*\innLen-2*\leiLen)}) (\sysLen-\innSep-\innLen,{(\sysHei-2*\innSep)/4*\x+\innSep}) to ++(-\leiLen,0);
  \draw[white] (\picLen,0) to ++(0,\sysHei);
\end{tikzpicture}

    &
    $S' > S$
  \end{tabular}
\end{table}

\uline{2. Hauptsatz}:\\
\fbox{\parbox{0.98\textwidth}{Es gibt eine Extensive Zustandsgröße, Entropie $S$, die im adiabatisch eingeschlossenen System nicht abnehmen kann: $\Delta S \geq 0$,}}\\
Häufig wird angenommen, dass $S$ eine streng monoton wachsende Funktion $S\norBra{E,X}$ der Energie und der extensiven Parameter $X$ ist, damit $S$ nach $E$ auflösbar ist: $E\norBra{S,X}$. Dies schließt negative Temperaturen aus.\\
Für quasistatische Prozesse: $S$ kann (wie alle anderen Zustandsgrößen) als Funktion eines Satzes von Zustandsvariablen geschrieben werden.\\
$\rightarrow$ $\dd{S}\norBra{E,X} = \pdv{S}{E}{X} \dd{E} + \pdv{S}{X}{E}\dd{X}$ ($\pdv{a}{b}{c} =$ partielle Ableitung bei festem $c$), $\frac{1}{T} \mDef \pdv{S}{E}{X}$\\
Für kontrolliert und reverersibel zugeführte Arbeit gilt: $\delta W = f\dd{X}$. Zusätzlich adiabatisch ($\delta Q =0$), dann gilt: $\dd{E}= \delta W = f\dd{X}$. Reversibel $+$ adiabatisch: $\dd{S} = 0$, $\dd{E} = f\dd{X}$ $\rightarrow$ $\frac{1}{T}f\dd{X} +\pdv{S}{X}{E}\dd{X} = 0$ $\rightarrow$ $\frac{f}{T} = -\pdv{S}{X}{E}$\\
\uwave{Alternativ}: Definiere $f$ über $\frac{f}{T} = -\pdv{S}{X}{E}$ und folgere daraus $\delta W = f\dd{X}$, wenn $\dd{S} = 0$ und $\delta Q= 0$.\\
$\left.\begin{array}{r l}\rightarrow & \dd{S} = \frac{\dd{E}}{T} - \frac{f}{T} \dd{X}\\ \text{Umstellen} \rightarrow & \text{\fbox{\dd{E} = T \dd{S} + f\dd{X}}}\end{array}\right\}\begin{array}{c}\text{\uline{Gibbsche}}\\\text{\uline{Fundamentalform}}\end{array}$\\
Für reversible Prozesse mit Wärmeaustausch gilt: $\dd{E} = \delta Q + f\dd{X}$\\
$\rightarrow$ \fbox{$\dd{S} = \frac{\delta Q}{T}$} ($\frac{1}{T}$ ist integrierender Faktor.)\\
Adiabatische Arbeit $\delta W_{nr}$, die nicht die Form $f\dd{X}$ hat: $\dd{S} \geq 0$, $\dd{E} = \delta W_{nr}$, $\dd{X} = 0$ $\rightarrow$ $\delta W_{nr} \geq 0$. Das heißt falls $\delta W_{nr} \neq 0$, dann $\dd{S} > 0$ (irreversibel) und $\delta W_{nr} > 0$ ($\nexists$ Perpetuum mobile 2. Art). Nichtreversibler Prozess mit Wärmeaustausch: $\dd{S} > \frac{\delta Q}{T}$.\\
$\rightarrow$ Formulierung des 2. Hauptsatzes für Prozesse mit Wärme: \fbox{$\dd{S} \geq \frac{\delta Q}{T}$}.\\
Ein allgemeiner Prozess mit konstanter Entropie heißt \uline{isentrop}.\\
\uwave{Beispiel}: Gas mit Energie $E$, Volumen $V$.\\
$\rightarrow$ $\dd{E} = T \dd{S} - p\dd{V}$ beziehungsweise $\dd{S} = \frac{\dd{E}}{T} + \frac{p}{T}\dd{V}$\\
mit Druck $p = T\cdot\pdv{S}{V}{E}$\\
Ideales Gas: $S = \frac52 N k_B + N k_B \ln\edgBra{\frac{V}{N h^3} \norBra{2 \pi m k_B T}^{\frac32}} = \frac52 N k_B + N k_B \ln\edgBra{\frac{V}{N h^3} \norBra{2 \pi m \frac{2 E}{3 V}}^{\frac32}}$\\
$\rightarrow$ $\pdv{S}{V}{E} = \frac{N k_B}{V}$ $\rightarrow$ \fbox{$p V = N k_B T$} \uline{thermische Zustandsgleichung}\\
\uline{Wärmekapazitäten}: $C=$\glqq$\frac{\delta Q}{\dd{T}}$\grqq$ = \frac{T\dd{S}}{\dd{T}} = T \ppv{S}{T}$\\
konstantes Volumen: $C_V = T \pdv{S}{T}{V} = \pdv{E}{T}{V}$; konstanter Druck: $C_p = T\pdv{S}{T}{p} \neq \pdv{E}{T}{p}$; Ideales Gas: $C_V = \frac32 N k_B$; $C_p = \frac52 N k_B$\\
Betrachte im Folgenden System mit extenxiven Variablen $V$, $N$ (\glqq einkomponentiges Fluidum\grqq). $\rightarrow$ $S = S\norBra{E,V,N}$.\\
Auflösen nach $E$ $\rightarrow$ $E = E\norBra{S,V,N}$\\
Totales Differential: $\dd{E} = \ppv{E}{S} \dd{S} + \ppv{E}{V} \dd{V} + \ppv{E}{N}\dd{N} = T \dd{S} -p \dd{V} +\mu\dd{N}$ da $\ppv{E}{S} = \frac{1}{\ppv{S}{E}} = T$; $\ppv{E}{V} = - p$ (Druck); $\ppv{E}{N} = \mu$ (chemisches Potential)\\
\uwave{Einschub}: Rechenregeln für partielle Ableitungen\\
Betrachte $z\norBra{x,y}$, das nach $x$ oder $y$ aufgelöst werden kann. $\rightarrow$ $x\norBra{y,z}$, $y\norBra{z,x}$

\begin{itemize}
  \item[1)] Bei festgehaltenem $y$ ist $x\norBra{y,z}$ die Umkehrfunktion von $z\norBra{x,y}$. $\rightarrow$ \fbox{$\pdv{x}{z}{y} = \frac{1}{\pdv{z}{x}{y}}$}
  \item[2)] $\left. \begin{array}{c}\dd{z} = \pdv{z}{x}{y}\dd{y} + \pdv{z}{y}{x}\dd{y}\\ \dd{x} = \pdv{x}{y}{z}\dd{y} + \pdv{x}{z}{y}\dd{z}\end{array}\right\} \stackrel[\text{Spezialfall}]{\dd{z} =0}{\longrightarrow} \begin{array}{c} \pdv{z}{x}{y}\dd{x} +\pdv{z}{y}{x}\dd{y} = 0\\ \dd{x} = \pdv{x}{y}{z}\dd{y} = 0\end{array}$\\
  $\rightarrow$ $\pdv{z}{x}{y}\pdv{x}{y}{z} + \pdv{z}{y}{x} = 0$\\
  $\rightarrow$ \fbox{$\pdv{z}{x}{y}\pdv{x}{y}{z}\pdv{y}{z}{x} = -1$}
\end{itemize}

$\rightarrow$ \fbox{$\dd{E} = T\dd{S} -p\dd{V}+\mu\dd{N}$} oder \fbox{$\dd{S} = \frac{1}{T}\dd{E} +\frac{p}{T}\dd{V}-\frac{\mu}{T}\dd{N}$}\\
Definition der Temperatur $T = \frac{1}{\ppv{S}{E}}$ stimmt mit der statistischen Physik überein.\\
Dort wurde gezeigt, dass für $2$ Systeme $a$, $b$ im thermischen Gleichgewicht gilt: $T_a= T_b$. (Wegen Entropiemaximierung $0=\ppv{}{E_a}\norBra{S_a+S_b} = \ppv{S_a}{E_a} - \ppv{S_b}{E_b}$)\\
$\rightarrow$ Temperaturdefinition erfüllt auch die Anforderungen an eine \uline{empirische Temperatur} (siehe \ref{sec:Temperatur5}).\\
\uline{Weitere Arten von Gleichgewicht}:

\begin{itemize}[align=left]
  \item[--] Zwei Systeme, die Volumen austauschen (das heißt mechanische Wechselwirkung) und Wärme austauschen (thermische Wechselwirkung):\\
  {\centering
  \begin{tikzpicture}[baseline=0.4cm]
  \def\sysLen{3};
  \def\sysHei{1};
  \def\filCol{white!90!black};
  \def\filTwo{white!60!black};
  \def\blaKkk{black};
  \def\borDis{0.1};
  \def\smaArr{0.5};
  \def\arrDis{0.2};
  \def\befFil{0.3};
  \fill[\filCol] (0,0) rectangle ++(\befFil*\sysLen,\sysHei);
  \fill[\filTwo] (\befFil*\sysLen,0) rectangle (\sysLen,\sysHei);
  \node at (\befFil*\sysLen/2,\sysHei/2) {$a$};
  \node at ({\sysLen-(\sysLen-\sysLen*\befFil)/2},\sysHei/2) {$b$};
  \draw (0,0) rectangle ++(\sysLen,\sysHei);
  \draw (-\borDis,-\borDis) rectangle ++ (\sysLen+2*\borDis,\sysHei+2*\borDis);
  \fill[\blaKkk] (\befFil*\sysLen,0) rectangle ++(\borDis,\sysHei);
  \draw[<->] (\befFil*\sysLen+\borDis/2-\smaArr/2,\sysHei/2) to ++(\smaArr,0);
\end{tikzpicture}

  }\\
  $V = V_a + V_b$, $S = S_a + S_b$, $T_a = T_b$\\
  $S$ ist maximal $\rightarrow$ $0 = \ppv{S}{V_a} = \ppv{S_a}{V_a} - \ppv{S_b}{V_b}$\\
  $\rightarrow$ $\frac{p_a}{T_a} = \frac{p_b}{T_b}$ $\rightarrow$ $p_a = p_b$
  \item[--] Zwei Systeme, die Wärme und Teilchen austauschen:\\
  $T_a = T_b$, $N = N_a + N_b$, $S = S_a + S_b$ $\rightarrow$ $\mu_a = \mu_b$
  \item[Beachte:]
  \begin{itemize}
      \item[--] Volumenaustausch ohne Energieänderung physikalisch wenig sinnvoll, außer wenn $p_b = 0$!
      \item[--] Bei Volumenaustausch ohne Wärme, aber mit Zulassen irreversibler Arbeit ($\delta W \neq - p \dd{V}$):\\
      gleiche Schlussfolgerung $p_a = p_b$
  \end{itemize}
\end{itemize}

\uline{Entropieänderung bei Temperaturausgleich}:\\
\begin{tikzpicture}[baseline=0.5cm]
  \def\boxLen{1};
  \def\boxDis{1};
  \def\stkLen{0.3};
  \def\stkAng{20};
  \draw (0,0) rectangle ++(\boxLen,\boxLen);
  \node at (\boxLen/2,2*\boxLen/3) {$a$};
  \node at (\boxLen/2,\boxLen/3) {$T_a$};
  \draw (\boxLen,\boxLen/2) to ++(\stkLen,0) to ++(\stkAng:\boxDis-2*\stkLen) (\boxLen+\boxDis,\boxLen/2) to ++(-\stkLen,0);
  \tikzset{shift={(\boxLen+\boxDis,0)}}
  \draw (0,0) rectangle ++(\boxLen,\boxLen);
  \node at (\boxLen/2,2*\boxLen/3) {$b$};
  \node at (\boxLen/2,\boxLen/3) {$T_b$};
\end{tikzpicture}
 $T_a > T_b$

\begin{itemize}[align=left]
  \item[Wärmekontakt:] $\Delta S = S_a\norBra{E_a + \Delta E_a} + S_b\norBra{E_b + \Delta E_b} - S_a\norBra{E_a} - S_b\norBra{E_b}$
  \item[Differentiell:] $\dd{S} = \ppv{S_a}{E_a}\dd{E_a} + \ppv{S_b}{E_b}\dd{E_b} = \norBra{\frac{1}{T_a} - \frac{1}{T_b}}\dd{E_a}$\\
  $\dd{S} > 0$ (2. Hauptsatz) und $\frac{1}{T_a} - \frac{1}{T_b} < 0$
  \item[Falls $T_a = T_b$:] $\dd{S} = 0$ trotz $\dd{E} \neq 0$ (reversible Wärme).
\end{itemize}

\uwave{Analog}: Bei zwei Systemen im thermischen Gleichgewicht ($T_a = T_b$) dehnt sich das System mit höherem Druck aus ($\delta V_a >0$), wenn mechanische Kopplung erlaubt, und Teilchen fließen vom höheren zum niedrigeren chemischen Potential, wenn Teilchenaustausch erlaubt.
\uline{Prinzip der minimalen Energie}\label{minEnergie}: Bei konstanter Entropie (sowie $V=\text{konstant}$, $N=\text{konstant}$) nimmt ein Systemden Zustand minimaler Energie ein (wennn interne Nebenbedingungen variiert werden können).

\begin{itemize}[align=left]
  \item[\uline{Beweis}:] \begin{itemize}[align=left]
    \item[1)] Betrachte Energieänderung $\delta E^{\norBra{1}}$ bei Einführen und Variation einer Nebenbedingung\\ \begin{tikzpicture}[baseline=0.5cm]
  \def\boxLen{3};
  \def\boxHei{1};
  \def\delPos{1};
  \def\delDif{0.5};
  \def\delLen{0.3};
  \def\boxCol{white!90!black};
  \draw[thick,fill=\boxCol] (0,0) rectangle ++(\boxLen,\boxHei);
  \draw[dashed,thick] (\delPos,0) to ++(0,\boxHei+\delLen);
  \draw[thick] (\delPos+\delDif,0) to ++(0,\boxHei+\delLen);
  \draw[<-] (\delPos,\boxHei+\delLen/2) to ++(\delDif,0);
  \node at (\delPos+\delDif/2,\boxHei+\delLen) {$\delta E$};
\end{tikzpicture}

    \item[2)] Enthemmung bei abgeschlossenem System $\rightarrow$ $\delta S^{\norBra{2}}>0$.
    \item[3)] Rückführung in den Anfangszustand durch Entnahme von Wärme: $\delta Q^{\norBra{3}}=-t\dd{S}^{\norBra{2}} < 0$ $\rightarrow$ $\delta E^{\norBra{1}} = -\delta Q^{\norBra{3}} > 0$.
\end{itemize}
  \item[\uline{Beachte}:] In $\dd{E} = T \dd{S} - p \dd{V} + \mu \dd{N}$ kommen nur Zustandsgrößen vor. diese sind zunächst nicht mit den Begriffen Arbeit und Wärme verbunden. Der Zusammenhang ist nicht universell, sondern wird erst durch die Art der Prozessrealisierung festgelegt.
  \item[\uline{Beispiel}:] $2$ Arten der Realisierung der quasistatischen Ausdehnung bei konstanter Energie
\end{itemize}

\renewcommand{\arraystretch}{1}
\setlength{\tabcolsep}{1pt}
\begin{table}[H]
  \centering
  \begin{tabular}{c | c}
    \uline{1) Adiabatisch eingeschlossen}
    &
    \underline{2) Mittels \glqq Wärmebad\grqq und \glqq Arbeitsbad\grqq }\\
     & ($=$ Reservoir an Wärme/ Arbeit)\\
    \begin{tikzpicture}[baseline=0.5cm]
  \def\sysLen{3};
  \def\sysHei{1};
  \def\pinRad{0.1};
  \def\pinDis{0.2};
  \def\pinExt{0.9};
  \def\filCol{white!90!black};
  \def\borDis{0.1};
  \def\numPin{7};
  \def\pinLef{3};
  \def\braDis{0.5};
  \def\braAmp{0.1};
  \fill[\filCol] (0,0) rectangle (\sysLen/2,\sysHei);
  \draw[thick] (0,0) rectangle ++(\sysLen,\sysHei);
  \draw[thick] (-\borDis,-\borDis) rectangle (\sysLen+\borDis,\sysHei+\borDis);

  \foreach \x in {1,...,\pinLef} \draw[thick] ({\sysLen/2-\x*\sysLen/2/(\numPin+1)},-\pinDis-\pinExt) circle (\pinRad) to ++(0,\sysHei+\pinDis);

  \draw[thick,decorate,decoration={brace,amplitude=\braAmp cm}] (0,\sysHei+\borDis) to node[above] {$V_a$} ({\sysLen/2-\pinLef*\sysLen/2/(\numPin+1)},\sysHei+\borDis);

  \draw[thick,dashed] (0,\sysHei+\borDis) to ++(0,\braDis);
  \draw[thick,decorate,decoration={brace,amplitude=\braAmp cm}] (0,\sysHei+\borDis+\braDis) to node[above] {$V_a$} ++(\sysLen,0);
  \draw[thick,dashed] (\sysLen,\sysHei+\borDis+\braDis) to ++(0,-\braDis);

  \foreach \x in {0,...,\numPin} \draw[thick] ({\sysLen/2+\x*\sysLen/2/(\numPin+1)},-\pinDis) circle (\pinRad) to ++(0,\sysHei+\pinDis);
\end{tikzpicture}
 & \begin{tikzpicture}[baseline=0.5cm]
  \def\sysLen{3};
  \def\sysHei{1};
  \def\sysDis{0.2};
  \def\syoLen{1};
  \def\filCol{white!90!black};
  \def\borDis{0.1};
  \def\ovrLap{0.2};

  \fill[\filCol] (0,0) rectangle (\sysLen/2,\sysHei);

  \draw[thick] (0,0) rectangle ++(\sysLen,\sysHei);

  \fill[black] (\sysLen/2-\borDis/2,0) rectangle ++(\borDis,\sysHei);

  \draw[thick] (-\sysDis-\syoLen,0) rectangle ++(\syoLen,\sysHei);

  \node at (-\sysDis-\syoLen/2,\sysHei/2) {$\delta Q$};

  \draw[thick] (\sysLen+\sysDis,0) rectangle ++(\syoLen,\sysHei);

  \node at (\sysLen+\sysDis+\syoLen/2,\sysHei/2) {$\delta W$};

  \draw[->] (-\sysDis-\ovrLap,\sysHei/2) to (\ovrLap,\sysHei/2);

  \draw[thick] (\sysLen/2,\sysHei/2) to (\sysLen+\sysDis+\ovrLap,\sysHei/2);

\end{tikzpicture}
\\
    \multicolumn{2}{c}{\tikzset{every node/.style={scale=0.8}}
\begin{tikzpicture}
  \draw[->] (0,-0.15) to (0,5.25);
  \draw[->] (-0.15,0) to (10.25,0);
  \node[anchor=south] at (0,5.35) {$S$};
  \node[anchor=west] at (10.35,0) {$V$};

  \draw (3,-0.15) to ++(0,0.3);
  \node[anchor=north] at (3,-0.25) {$V_a$};
  \draw (7,-0.15) to ++(0,0.3);
  \node[anchor=north] at (7,-0.25) {$V_e$};

  \draw[thick] (3,1) .. controls (3.5,2) and (6.5,4) .. (7,4);

  \fill[black] (3,1) circle (0.1) (7,4) circle (0.1);

  \draw[thick,dashed] (3,1) to (7,1);

  \draw[decorate,decoration={brace,amplitude=0.1 cm}] (7,4) to node[right] {$\Delta S>0$} (7,1);

\end{tikzpicture}
} \\
    $\delta Q = \delta W = 0$ & Zu jedem Zeitpunkt:\\
    $\Delta S_{\text{ges}} = \Delta S > 0$ & $\dd{E} = \delta{Q} + \delta W = 0$\\
    (\uline{irreversibel}) & $\delta W = -p \dd{V}$\\
    $\Delta S = \int\limits_{V_a}^{V_e} \pdv{S}{V}{E} \dd{V} = \int\limits_{V_a}^{V_e} \frac{p}{T} \dd{V}$ & $\dd{E} = T\dd{S} -p\dd{V}$\\
     & $\rightarrow$ $\dd{S} = -\frac{\delta Q_{\text{res}}}{T}=-\dd{S_\text{res}}$\\
     & (Temperatur des Wärmebads so zu wählen, dass $T_\text{res} = T\norBra{E_a,V}$)\\
     & $\rightarrow$ $\Delta S = \int\dd{S} = \int\limits_{V_a}^{V_e} \frac{p}{T} \dd{V} = - \int\dd{S_\text{res}} = -\Delta S_\text{res}$\\
     & $\rightarrow$ $\Delta S_\text{ges} = 0$ (\uline{reversibel})
  \end{tabular}
\end{table}

\uline{Zusammenfassung Entropie}

\begin{itemize}[align=left]
  \item[--] Natürlich ablaufende Prozesse  (Streben ins Gleichgewicht) sind irreversibel und bedeuten $\Delta S > 0$ $\eDef$ Entropiemaximierung der statistischen Physik
  \item[--] Quasistatische Prozsse erfüllen $\dd{E} = T\dd{S}-p\dd{V}+\mu\dd{N}$ unabhängig von der Prozessrealisierung (Arbeit/ Wärme)
  \item[--] Temperaturdefinition $T^{-1} = \pdv{S}{E}{V,N}$ $\rightarrow$ Gleichgewichtsbedingung $T_a = T_b$.
  \item[Achtung:] Gelegentlich wird \glqq quasistatisch\grqq\ (anders als hier) im Sinne von \glqq reversibel\grqq\ benutzt.
\end{itemize}

\subsection{Kreisprozesse}
\uline{Ziel}:
\begin{itemize}[align=left]
  \item[--] Verrichten von Arbeit durch Umwandlung von Wärme in Arbeit (\uline{Wärme\-kraft\-maschine}/ \uline{Arbeits\-masch\-ine})
  \item[--] Transport von Wärme durch Arbeitseinsatz (\uline{Wärmepumpe}, \uline{Kältemaschine})
\end{itemize}
\uline{Maximale Arbeit}
\begin{itemize}[align=left]
  \item[Maschine:]
  \begin{itemize}[align=left]
      \item[\begin{tikzpicture}[baseline=0.5cm]
  \def\sysLen{3};
  \def\sysHei{1};
  \def\sysDis{0.5};
  \def\arrDis{0.2};
  \def\arrLen{0.5};
  \def\borDis{0.1};
  \def\syoLen{1};
  \def\pistIn{0.3};

  \draw (0,0) rectangle (\sysLen,\sysHei);

  \draw (-\borDis,-\borDis) rectangle (\sysLen+\borDis,\sysHei+\borDis);

  \draw (\sysLen+\sysDis,0) rectangle ++(\syoLen,\sysHei);

  \draw (\sysLen-\pistIn,\sysHei/2) to (\sysLen+\sysDis+\pistIn,\sysHei/2);

  \fill[black] (\sysLen+\sysDis+\pistIn-\borDis/2,0) rectangle ++(\borDis,\sysHei);

  \draw[<->] (\sysLen+\sysDis+\pistIn-\arrLen/2,\sysHei/2+\arrDis) to ++(\arrLen,0);

  \node[anchor=south] at (\sysLen+\sysDis+\pistIn,\sysHei+\borDis) {$W$};
\end{tikzpicture}
]
      \item[Anfang] $S_0$, $E_o\norBra{S_0}$
      \item[Ende] $S$, $E\norBra{S}$
      \item[(äußere Parameter eventuell unterschiedlich)]
    \end{itemize}
  \item[$\rightarrow$] $W=E\norBra{S} - E_0\norBra{S_0} = W\norBra{S,S_0}$
  \item[Arbeit als  Funktion der Entropie:] $\ppv{W}{S} = \ppv{E}{S} = T > 0$
  \item[Von der Maschine verrichtete Nutz-Arbeit:] $W_N = -W$, $\ppv{W_N}{S} < 0$
  \item[$\rightarrow$] Entropie muss möglichst klein sein, damit $W_N$ maximal.
  \item[Idealfall:] $S=S_0$
\end{itemize}

\uline{Carnot-Prozess}\\
Kombination aus isothermen (konstante Temperatur) und isentropen Prozessen an einem Hilfssystem (Arbeitsgas, Teil der Maschine). $\Delta Q_1 <0$ $\Delta Q_2 > 0$

\tikzset{every node/.style={scale=0.8}}
\begin{figure}[H]
  \centering
  \tikzset{every node/.style={scale=0.8}}
\begin{tikzpicture}
  \draw[->] (0,-0.15) to (0,5.25);
  \draw[->] (-0.15,0) to (10.25,0);
  \node[anchor=south] at (0,5.35) {$T$};
  \node[anchor=west] at (10.35,0) {$S$};

  \draw (2,-0.15) to ++(0,0.3);
  \draw (8,-0.15) to ++(0,0.3);
  \node[anchor=north] at (5,-0.25) {$\verBra{\Delta S_{1/2}}$};

  \draw (-0.15,1) to ++(0.3,0);
  \node[anchor=north] at (-0.25,1) {$T_1$};
  \draw (-0.15,4) to ++(0.3,0);
  \node[anchor=north] at (-0.25,4) {$T_2$};

  \draw[thick] (2,1) to (2,4) to (8,4) to (8,1) -- cycle;

  \draw[thick,->] (2,1) to (2,2.5);
  \draw[thick,->] (2,4) to (5,4);
  \draw[thick,->] (8,4) to (8,2.5);
  \draw[thick,->] (8,1) to (5,1);

  \draw[thick,->] (7,4.5) to (7,3.5);
  \node[anchor=west] at (7.1,4.5) {$\Delta Q_2$};
  \draw[thick,->] (7,1.5) to (7,0.5);
  \node[anchor=west] at (7.1,1.5) {$\Delta Q_1$};

\end{tikzpicture}

\end{figure}
\begin{figure}[H]
  \centering
  \tikzset{every node/.style={scale=0.8}}
\begin{tikzpicture}
  \draw[->] (0,-0.15) to (0,5.25);
  \draw[->] (-0.15,0) to (10.25,0);
  \node[anchor=south] at (0,5.35) {$T$};
  \node[anchor=west] at (10.35,0) {$V$};

  \draw[thick] (4,1) .. controls (3.4,1.5) and (3.4,1.5) .. (3,2) .. controls (2.6,2.5) and (2.6,2.5) .. (2,4) to (6,4) .. controls (6.6,2.5) and (6.6,2.5) .. (7,2) .. controls (7.4,1.5) and (7.4,1.5) .. (8,1) -- cycle;

  \draw[thick,->] (4,1) .. controls (3.4,1.5)  .. (3,2);
  \draw[thick,->] (2,4) to (4,4);
  \draw[thick,->] (6,4) .. controls (6.6,2.5) .. (7,2);
  \draw[thick,->] (8,1) -- (6,1);;

  \draw[thick,->] (3,4.5) to (3,3.5);
  \node[anchor=west] at (3.1,4.5) {$\Delta Q_2$};
  \draw[thick,->] (5,1.5) to (5,0.5);
  \node[anchor=west] at (5.1,1.5) {$\Delta Q_1$};
\end{tikzpicture}

\end{figure}
\begin{itemize}[align=left]
\item[Hilfssysteme arbeitet zyklisch und reversibel:] $\Delta  S = \frac{\Delta Q_1}{T_1} + \frac{\Delta Q_2}{T_2} = 0$\\ $\rightarrow$ $\Delta Q_1 = - \frac{T_1}{T_2}\Delta Q_2$
\item[Gewonnene Arbeit:] $W_N = \Delta Q_2 + \Delta Q_1 = T_2 \Delta S_2 - T_1 \Delta S_2 = \norBra{T_2-T_1}\Delta S_2$\\ $=$ Fläche im $T$-$S$-Diagramm; $W_N = \norBra{1-\frac{T_1}{T_2}}\Delta Q_2$
\item[\uline{Wirkungsgrad der Carnot-Wärmekraftmaschine}:] \fbox{$\eta = \frac{W_N}{\Delta Q_2} = 1-\frac{T_1}{T_2}$}
\item[Damit $W_N > 0$:] Rechtsprozeß (Umlauf im Uhrzeigersinn), $\Delta Q_1 < 0$
\item[Linksprozeß (gegen Uhrzeigersinn):] $\Delta Q_2 <0$, $\Delta Q_1 > 0$
\item[$\rightarrow$] Betrieb als Wärmepumpe, Wirkungsgrad $\eta^H = \verBra{\frac{\Delta Q_2}{W_N}} = \frac{T_2}{T_2-T_1}$ (Heizeffektivität)
\item[$\rightarrow$] Betrieb als Kältemaschine, Wirkungsgrad $\eta^K = \verBra{\frac{\Delta Q_1}{W_N}} = \frac{T_1}{T_2-T_1}$ (Kühleffektivität)
\end{itemize}
Carnot-Prozess dient vor allem als Idealisierung. In der Realität ist der isotherme Prozess nicht sehr praktikabel für einen Motor (lange Wartezeiten) und es treten Reibungsverluste auf ($\Delta S_\text{ges} > 0$).

\uline{Allgemeiner Kreisprozess}
\begin{figure}[H]
  \centering
  \tikzset{every node/.style={scale=0.8}}
\begin{tikzpicture}
  \draw[->] (0,-0.15) to (0,5.25);
  \draw[->] (-0.15,0) to (10.25,0);
  \node[anchor=south] at (0,5.35) {$T$};
  \node[anchor=west] at (10.35,0) {$S$};

  \draw (-0.15,1) to ++(0.3,0);
  \draw[dashed] (0,1) to (10,1);
  \node[anchor=north] at (-0.25,1) {$T_1$};
  \draw (-0.15,4) to ++(0.3,0);
  \draw[dashed] (0,4) to (10,4);
  \node[anchor=north] at (-0.25,4) {$T_2$};

  \draw[thick] (3,1) .. controls (1,3) and (5,3) .. (5,4) .. controls (8,4) and (8,1) .. cycle;
  \draw[thick,->] (3,1) .. controls (1,3) and (5,3) .. (5,4);

  \draw[thick,->] (7,4.5) to (7,3.5);
  \node[anchor=west] at (7.1,4.5) {$\Delta Q_2$};
  \draw[thick,->] (7,1.5) to (7,0.5);
  \node[anchor=west] at (7.1,1.5) {$\Delta Q_1$};

\end{tikzpicture}

\end{figure}

$W_N= \Delta Q = \Delta Q_2 + \Delta Q_1$ mit $\Delta Q_2 > 0$, $\Delta Q_1 < 0$ beliebig auf dem Zyklus verteilt

\begin{itemize}[align=left]
  \item[Zweiter Hauptsatz:] $\dd{S} \geq \frac{\delta Q}{T}$ $\rightarrow$ $\Delta S = \oint \dd{S} \geq \oint \frac{\delta Q}{T} = \int\limits_{\delta Q > 0} \frac{\delta Q}{T} + \int\limits_{\delta Q < 0} \frac{\delta Q} {T}$
  \item[Zyklisch] $\rightarrow$ $\Delta S = 0$
  \item[Abschätzungen] Für $\delta Q > 0$: $\frac{\delta Q}{T} \geq\frac{\delta Q}{T_2}$, für $\delta Q < 0$: $\frac{\delta Q}{T} \geq \frac{\delta Q}{T_1}$
  \item[$\rightarrow$] $0 \geq \frac{1}{T_2} \int\limits_{\delta Q > 0} \delta Q + \frac{1}{T_1} \int\limits_{\delta Q  < 0} \delta Q = \frac{\Delta Q_2}{T_2} +\frac{\Delta Q_1}{T_1}$
  \item[$\rightarrow$] $\frac{\Delta Q_1}{\Delta Q_2} \leq - \frac{T_1}{T_2}$
  \item[$\rightarrow$] $W_N = \Delta Q_2\norBra{1 + \frac{\Delta Q_1}{\Delta Q_2}} \leq \Delta Q_2\norBra{1- \frac{T_1}{T_2}}$
  \item[$\rightarrow$] Wirkungsgrad \fbox{$\eta \leq 1 - \frac{T_1}{T_2}$}
\end{itemize}

Idealer Wirkungsgrad wird nur erreicht, wenn keine irreversiblen Prozesse und wenn Wärmeüberträge bei Minimal- und Maximaltemperatur.
