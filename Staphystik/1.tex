\section{Einleitung}
\begin{tabbing}
Statistische Mechanik oder allgemeiner statistische Physik\\
$=$ Beschreibung von Systemen mit großer Teilchenzahl\\
Wesentliche Elemente:\\
\hspace{4em} \= \kill
--\> Auswahl einiger makroskopischer Variablen, die den Systemzustand ausreichend charakterisieren\\
\> (Druck, Temperatur, \dots)\\
--\> Fehlende Informationen über mikroskopische Details\\
$\rightarrow$\> wahrscheinlichkeitsbasierte Theorie\\
Historisch entstanden aus der Thermodynamik (Theorie der Wärme)\\
\underline{Geschichte}:\\
$\sim$ \= 1660: Boyle-Mariotte-Gesetz ($p V =$ const)\\
\> 1738: Bernoulli, Beginn der kinetischen Gastheorie, mikroskopische Deutung von Druck und Temperatur.\\
\> 1824: Carnot, Begriff der Arbeit, Beginn der modernen Thermodynamik\\
\> 1843: Joule, Zusammenhang Wärme-Arbeit, Energieerhaltung\\
\> 1850: Clausius, Begriff der Entropie\\
\> 1859: Maxwellverteilung\\
\> 1875: Boltzmann, mikroskopische Erklärung der Entropie\\
\> 1902: Gibbs, detaillierte Formulierung der statistischen Mechanik (zum Beispiel kanonisches Ensemble)\\
\> 1932: von Neumann-Entropie
\end{tabbing}
