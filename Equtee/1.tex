\section{Wellenmechanik}
\subsection{Klassische Physik}

\subsubsection{(Hamiltonsche) Mechanik}
\begin{itemize}
  \item Der \Hl{Zustand} eines Punktteilchens wird durch die Angabe von \Hl{Ort} $\vX$ und \Hl{Impuls} $\vP$, das heißt durch einen Punkt im Phasenraum beschrieben
  \item Seine Dynamik ist durch die \Hl{Hamiltonfunktion} $H\of{\vX,\vP}$ und die \Hl{Hamiltonschen Be\-we\-gungs\-gleich\-ung\-en}
  \begin{equation}
    \dot{x}_i = \ppv{H}{p_i} \quad \dot{p}_i = - \ppv{H}{x_i}
  \end{equation}
  festgelegt.\\
  Zum Beispiel für ein Teilchen der Masse $m$ in einem zeitabhängigen Potential $V\of{\vX}$ ist die Hamiltonfunktion die Gesamtenergie
  \begin{equation}
    H\of{\vX,\vP} = \frac{\vP^2}{2m} + V\of{\vX}
  \end{equation}
  und die Bewegungsgleichungen
  \begin{equation}
    \dot{x}_i = \frac{\dot{p}_i}{m} = v_i \quad \dot{p}_i = - \ppv{V}{x_i} = F_i
  \end{equation}
  \item Lösen der Bewegungsgleichungen unter Anfangsbedingungen $\norBra{\vX_0,\vP_0}$ zum Zeitpunkt $t_0$ ergibt die \Hl{Trajektorie} $\norBra{\vX\of{t},\vP\of{t}}$ des Teilchens. Die Trajektorie legt die Zustände des Teilchens zu beliebigen anderen Zeitpunkten fest.
  \item Die Messung einer physikalischen \Hl{Messgröße} $M\of{\vX,\vP}$ zum Zeitpunkt $t$ liefert wie erwartet das Ergebnis $M\of{\vX\of{t},\vP\of{t}}$. Zum Beispiel Energie $H\of{\vX,\vP}$, Drehimpuls $\vL = \vX\times\vP$
\end{itemize}
Alle diese Punkte müssen in der Quantenmechanik revidiert werden.

\subsubsection{Klassische Elektrodynamik}
\begin{itemize}
  \item Elektromagnetische Felder werden durch Angabe des elektrischen Feldes $\vE\of{\vX}$ und magnetischen Feldes $\vB\of{\vX}$ beschrieben.
  \item Deren Dynamik ist durch die Maxwellgleichungen festgelegt. In Abwesenheit von Ladung und Strömen, erfüllen die Felder die Wellengleichung
    \begin{equation}
      \norBra{\frac{1}{c^2}\nppv{}{t}{2} - \Laplace} \E\of{\vX,t} = 0
    \end{equation}
  mit der fundamentalen Lösung
    \begin{equation}
      \E\of{\vX,t} = A \e{\mI\norBra{\vK\vX- \omega t}}
    \end{equation}
  wobei $\E$ eine beliebige Komponente von $\vE\of{\vX}$ darstellt und $A$ die \Hl{Amplitude} der Welle bezeichnet. Die physikalische Lösung ist $\Re\of{\E\of{\vX,t}} = \frac{1}{2}\norBra{\E\of{\vX,t} + \E\of{\vX,t}^*}$.
  \item Die \Hl{Dispersionsrelation} das heißt der Zusammenhang zwischen \Hl{Kreisfrequenz} $\omega = 2 \pi \nu$ und \Hl{Wel\-len\-vek\-tor} $\vK$ beziehungsweise \Hl{Wellenzahl} $k = \Norm{\vK}$, lautet für elektromagnetische Wellen (im Vakuum)
  \begin{equation}
    \vK^2 - \frac{\omega^2}{c^2} = 0 \quad \text{beziehungsweise} \quad k = \Norm{\vK} = \frac{\omega}{c}
  \end{equation}
  \item Die Wellengleichung ist linear, also gilt das \Hl{Superpositionsprinzip}: Jede Linearkombination von Lösungen ist wieder eine Lösung. Die Allgemeine Lösung ist eine Superposition
  \begin{equation}
    \E\of{\vX,t} = \int\dn{k}{3} A\of{\vK} \e{\mI\norBra{\vK\vX - \omega t}}
  \end{equation}
  wobei $A\of{\vK}$ die Amplitude für Komponente mit Wellenvektor $\vK$ ist.
  \item Die Energie im elektromagnetischen Feld ist
  \begin{align}
    E & = \int \dn{x}{3} \frac{\epsilon_0}{2}\norBra{\E^2\of{\vX,t}+ c^2\B^2\of{\vX,t}} \\
      & = \int \dn{x}{3} \epsilon_0 \E^2\of{\vX,t}
  \end{align}
  (wegen $\Norm{\E} = c\Norm{\B}$) und der Energiefluss (Poynting-Vektor)
  \begin{equation}
    \vS =\mu_0 \vE\times\vB
  \end{equation}
  beziehungsweise für skalare Felder
  \begin{equation}
    \Norm{\vS} = c \epsilon_0 \E^2\of{\vX,t}
  \end{equation}
  Gemessen wird (in der Regel) die Intensität. Das ist der zeitlich über eine Periode $T = \frac{2\pi}{\omega} = \frac{1}{\nu}$ gemittelte Energiefluss
  \begin{align}
    I\of{\vX,t} & = c \epsilon_0 \frac{1}{T} \int\limits_t^{t+T} \dd{t'} \E^2\of{\vX,t} \\
               & = c \epsilon_0 \Norm{\E\of{\vX,t}}^2
  \end{align}
  \item Der Nachweis der Wellennatur des elektromagnetischen Feldes im Youngschen Doppelspaltexperiment (Abb. Inspiriert aus Schwabl, Quantenmechanik)
\end{itemize}

\begin{figure}[H]
  \centering
  \tikzset{thick,every node/.style={scale=1}}
\begin{tikzpicture}
  \def\wavFrt{3};
  \def\wavArr{3};
  \def\arrAng{60};
  \def\arrLen{1.7};
  \def\frtDis{0.7};
  \def\sliDis{2.5};
  \def\slitEx{0.8};
  \def\openSl{0.3};
  \def\blndEx{1.4};
  \def\scrnDs{4.5};
  \def\scdWav{2};
  \def\scdFDs{0.3};
  \def\scdArr{2};
  \def\scdAng{100};
  \def\scArLn{0.8};
  \def\lensLn{0.5};
  \def\lenThk{0.1};
  \def\imgDis{1};
  \def\pltThk{0.05};
  \def\pltStr{0.8};
  \def\pltIth{0.01};
  \def\lamDee{2.1};
  \tikzmath{\scrnEx = \slitEx+\openSl+\blndEx;
  \slitMi = \slitEx+\openSl/2;}

  \fill (0,0) circle(0.1);
  \foreach \x in {1,...,\wavFrt} \draw (0,-\x*\frtDis) arc(-90:90:\x*\frtDis);
  \foreach \x in {-\wavArr,...,\wavArr} \draw[->] (0,0) to ++(\x*\arrAng/\wavArr:\arrLen);
  \node[anchor=south] at (\sliDis,\scrnEx+0.1) {Doppelspalt};
  \node[anchor=south] at (\scrnDs,\scrnEx+0.1) {Schirm};
  \node[anchor=north] at (\scrnDs,-\scrnEx-0.1) {Intensität $I_1\of{\vX}$};
  \node[anchor=south west] at (0,\scrnEx+0.1) {Quelle};
  \node[anchor=north] at (\sliDis,-\scrnEx-0.1) {(a)};
  \draw (\sliDis,-\slitEx-\openSl-\blndEx) to ++(0,\blndEx) ++(0,\openSl) to ++(0,2*\slitEx) ++(0,\openSl) to ++(0,\blndEx);
  \node[anchor=south east] at (\sliDis-0.1,\slitMi+0.1) {$1$};
  \node[anchor=north east] at (\sliDis-0.1,-\slitMi-0.1) {$2$};
  \draw (0,0) to ++(\sliDis,\slitEx+\openSl/2);
  \foreach \x in {-\scdArr,...,\scdArr} \draw[->] (\sliDis,\slitEx+\openSl/2) to ++(\x*\scdAng/\wavArr:\scArLn);
  \foreach \x in {1,...,\scdWav} \draw (\sliDis,\slitEx+\openSl/2) ++(0,-\x*\scdFDs) arc(-90:90:\x*\scdFDs);
  \fill (\sliDis,-\slitMi-\lensLn/2) rectangle ++(\lenThk,\lensLn);

  \draw[fill=white!80!black,shift={(\scrnDs,0)},samples=100,domain=-\scrnEx:\scrnEx] (0,-\scrnEx) plot ({-\pltThk*((sin(180*sin(atan(\pltStr*(\x-\slitMi)))))/(sin(atan(\pltStr*(\x-\slitMi)))))^2},\x) -- (0,\scrnEx) -- cycle;

  \tikzset{shift={(\imgDis+\scrnDs,0)}}

  \fill (0,0) circle(0.1);
  \foreach \x in {1,...,\wavFrt} \draw (0,-\x*\frtDis) arc(-90:90:\x*\frtDis);
  \foreach \x in {-\wavArr,...,\wavArr} \draw[->] (0,0) to ++(\x*\arrAng/\wavArr:\arrLen);
  \node[anchor=south] at (\sliDis,\scrnEx+0.1) {Doppelspalt};
  \node[anchor=south] at (\scrnDs,\scrnEx+0.1) {Schirm};
  \node[anchor=north] at (\scrnDs,-\scrnEx-0.1) {Intensität $I_2\of{\vX}$};
  \node[anchor=south west] at (0,\scrnEx+0.1) {Quelle};
  \node[anchor=north] at (\sliDis,-\scrnEx-0.1) {(b)};
  \draw (\sliDis,-\slitEx-\openSl-\blndEx) to ++(0,\blndEx) ++(0,\openSl) to ++(0,2*\slitEx) ++(0,\openSl) to ++(0,\blndEx);
  \node[anchor=south east] at (\sliDis-0.1,\slitMi+0.1) {$1$};
  \node[anchor=north east] at (\sliDis-0.1,-\slitMi-0.1) {$2$};
  \draw (0,0) to ++(\sliDis,-\slitMi);
  \foreach \x in {-\scdArr,...,\scdArr} \draw[->] (\sliDis,-\slitMi) to ++(\x*\scdAng/\wavArr:\scArLn);
  \foreach \x in {1,...,\scdWav} \draw (\sliDis,-\slitMi) ++(0,-\x*\scdFDs) arc(-90:90:\x*\scdFDs);
  \fill (\sliDis,+\slitMi-\lensLn/2) rectangle ++(\lenThk,\lensLn);

  \draw[fill=white!80!black,shift={(\scrnDs,0)},samples=100,domain=-\scrnEx:\scrnEx] (0,-\scrnEx) plot ({-\pltThk*((sin(180*sin(atan(\pltStr*(\x+\slitMi)))))/(sin(atan(\pltStr*(\x+\slitMi)))))^2},\x) -- (0,\scrnEx) -- cycle;

  \tikzset{shift={(\imgDis+\scrnDs,0)}}

  \fill (0,0) circle(0.1);
  \foreach \x in {1,...,\wavFrt} \draw (0,-\x*\frtDis) arc(-90:90:\x*\frtDis);
  \foreach \x in {-\wavArr,...,\wavArr} \draw[->] (0,0) to ++(\x*\arrAng/\wavArr:\arrLen);
  \node[anchor=south] at (\sliDis,\scrnEx+0.1) {Doppelspalt};
  \node[anchor=south] at (\scrnDs,\scrnEx+0.1) {Schirm};
  \node[anchor=north] at (\scrnDs,-\scrnEx-0.1) {Intensität $I\of{\vX}$};
  \node[anchor=south west] at (0,\scrnEx+0.1) {Quelle};
  \node[anchor=north] at (\sliDis,-\scrnEx-0.1) {(c)};
  \draw (\sliDis,-\slitEx-\openSl-\blndEx) to ++(0,\blndEx) ++(0,\openSl) to ++(0,2*\slitEx) ++(0,\openSl) to ++(0,\blndEx);
  \node[anchor=south east] at (\sliDis-0.1,\slitMi+0.1) {$1$};
  \node[anchor=north east] at (\sliDis-0.1,-\slitMi-0.1) {$2$};
  \draw (0,0) to ++(\sliDis,-\slitMi);
  \foreach \x in {-\scdArr,...,\scdArr} \draw[->] (\sliDis,-\slitMi) to ++(\x*\scdAng/\wavArr:\scArLn);
  \foreach \x in {1,...,\scdWav} \draw (\sliDis,-\slitMi) ++(0,-\x*\scdFDs) arc(-90:90:\x*\scdFDs);
  \draw (0,0) to ++(\sliDis,\slitEx+\openSl/2);
  \foreach \x in {-\scdArr,...,\scdArr} \draw[->] (\sliDis,\slitEx+\openSl/2) to ++(\x*\scdAng/\wavArr:\scArLn);
  \foreach \x in {1,...,\scdWav} \draw (\sliDis,\slitEx+\openSl/2) ++(0,-\x*\scdFDs) arc(-90:90:\x*\scdFDs);

  \draw[fill=white!80!black,shift={(\scrnDs,0)},samples=100,domain=-\scrnEx:\scrnEx] (0,-\scrnEx) plot ({-\pltIth*((sin(360*\lamDee*sin(atan(\pltStr*\x))))/(sin(atan(\pltStr*\x))))^2},\x) -- (0,\scrnEx) -- cycle;

\end{tikzpicture}

  \caption{Beugung am Doppelspalt (a) mit Spalt $1$ geöffnet, (b) mit Spalt $2$ geöffnet, (c) beide Spalte geöffnet}
\end{figure}

\begin{align}
  I & \sim \Norm{E}^2 \\
    & \sim \Norm{E_1}^2 + \Norm{E_2}^2 + 2 \Re\of{E_1E_2^*}
\end{align}

\subsubsection{Zusammenfassung}

\begin{table}[H]
  \centering
  \tikzset{thick,every node/.style={scale=1}}
\begin{tikzpicture}
  \def\wavFrt{3};
  \def\wavArr{3};
  \def\arrAng{60};
  \def\arrLen{1.7};
  \def\frtDis{0.7};
  \def\sliDis{2.5};
  \def\slitEx{0.8};
  \def\openSl{0.3};
  \def\blndEx{1.4};
  \def\scrnDs{4.5};
  \def\scdWav{2};
  \def\scdFDs{0.3};
  \def\scdArr{2};
  \def\scdAng{100};
  \def\scArLn{0.8};
  \def\lensLn{0.5};
  \def\lenThk{0.1};
  \def\imgDis{1};
  \def\pltThk{0.05};
  \def\pltStr{0.8};
  \def\pltIth{0.01};
  \def\lamDee{2.1};
  \tikzmath{\scrnEx = \slitEx+\openSl+\blndEx;
  \slitMi = \slitEx+\openSl/2;}

  \fill (0,0) circle(0.1);
  \foreach \x in {1,...,\wavFrt} \draw (0,-\x*\frtDis) arc(-90:90:\x*\frtDis);
  \foreach \x in {-\wavArr,...,\wavArr} \draw[->] (0,0) to ++(\x*\arrAng/\wavArr:\arrLen);
  \node[anchor=south] at (\sliDis,\scrnEx+0.1) {Doppelspalt};
  \node[anchor=south] at (\scrnDs,\scrnEx+0.1) {Schirm};
  \node[anchor=north] at (\scrnDs,-\scrnEx-0.1) {Intensität $I_1\of{\vX}$};
  \node[anchor=south west] at (0,\scrnEx+0.1) {Quelle};
  \node[anchor=north] at (\sliDis,-\scrnEx-0.1) {(a)};
  \draw (\sliDis,-\slitEx-\openSl-\blndEx) to ++(0,\blndEx) ++(0,\openSl) to ++(0,2*\slitEx) ++(0,\openSl) to ++(0,\blndEx);
  \node[anchor=south east] at (\sliDis-0.1,\slitMi+0.1) {$1$};
  \node[anchor=north east] at (\sliDis-0.1,-\slitMi-0.1) {$2$};
  \draw (0,0) to ++(\sliDis,\slitEx+\openSl/2);
  \foreach \x in {-\scdArr,...,\scdArr} \draw[->] (\sliDis,\slitEx+\openSl/2) to ++(\x*\scdAng/\wavArr:\scArLn);
  \foreach \x in {1,...,\scdWav} \draw (\sliDis,\slitEx+\openSl/2) ++(0,-\x*\scdFDs) arc(-90:90:\x*\scdFDs);
  \fill (\sliDis,-\slitMi-\lensLn/2) rectangle ++(\lenThk,\lensLn);

  \draw[fill=white!80!black,shift={(\scrnDs,0)},samples=100,domain=-\scrnEx:\scrnEx] (0,-\scrnEx) plot ({-\pltThk*((sin(180*sin(atan(\pltStr*(\x-\slitMi)))))/(sin(atan(\pltStr*(\x-\slitMi)))))^2},\x) -- (0,\scrnEx) -- cycle;

  \tikzset{shift={(\imgDis+\scrnDs,0)}}

  \fill (0,0) circle(0.1);
  \foreach \x in {1,...,\wavFrt} \draw (0,-\x*\frtDis) arc(-90:90:\x*\frtDis);
  \foreach \x in {-\wavArr,...,\wavArr} \draw[->] (0,0) to ++(\x*\arrAng/\wavArr:\arrLen);
  \node[anchor=south] at (\sliDis,\scrnEx+0.1) {Doppelspalt};
  \node[anchor=south] at (\scrnDs,\scrnEx+0.1) {Schirm};
  \node[anchor=north] at (\scrnDs,-\scrnEx-0.1) {Intensität $I_2\of{\vX}$};
  \node[anchor=south west] at (0,\scrnEx+0.1) {Quelle};
  \node[anchor=north] at (\sliDis,-\scrnEx-0.1) {(b)};
  \draw (\sliDis,-\slitEx-\openSl-\blndEx) to ++(0,\blndEx) ++(0,\openSl) to ++(0,2*\slitEx) ++(0,\openSl) to ++(0,\blndEx);
  \node[anchor=south east] at (\sliDis-0.1,\slitMi+0.1) {$1$};
  \node[anchor=north east] at (\sliDis-0.1,-\slitMi-0.1) {$2$};
  \draw (0,0) to ++(\sliDis,-\slitMi);
  \foreach \x in {-\scdArr,...,\scdArr} \draw[->] (\sliDis,-\slitMi) to ++(\x*\scdAng/\wavArr:\scArLn);
  \foreach \x in {1,...,\scdWav} \draw (\sliDis,-\slitMi) ++(0,-\x*\scdFDs) arc(-90:90:\x*\scdFDs);
  \fill (\sliDis,+\slitMi-\lensLn/2) rectangle ++(\lenThk,\lensLn);

  \draw[fill=white!80!black,shift={(\scrnDs,0)},samples=100,domain=-\scrnEx:\scrnEx] (0,-\scrnEx) plot ({-\pltThk*((sin(180*sin(atan(\pltStr*(\x+\slitMi)))))/(sin(atan(\pltStr*(\x+\slitMi)))))^2},\x) -- (0,\scrnEx) -- cycle;

  \tikzset{shift={(\imgDis+\scrnDs,0)}}

  \fill (0,0) circle(0.1);
  \foreach \x in {1,...,\wavFrt} \draw (0,-\x*\frtDis) arc(-90:90:\x*\frtDis);
  \foreach \x in {-\wavArr,...,\wavArr} \draw[->] (0,0) to ++(\x*\arrAng/\wavArr:\arrLen);
  \node[anchor=south] at (\sliDis,\scrnEx+0.1) {Doppelspalt};
  \node[anchor=south] at (\scrnDs,\scrnEx+0.1) {Schirm};
  \node[anchor=north] at (\scrnDs,-\scrnEx-0.1) {Intensität $I\of{\vX}$};
  \node[anchor=south west] at (0,\scrnEx+0.1) {Quelle};
  \node[anchor=north] at (\sliDis,-\scrnEx-0.1) {(c)};
  \draw (\sliDis,-\slitEx-\openSl-\blndEx) to ++(0,\blndEx) ++(0,\openSl) to ++(0,2*\slitEx) ++(0,\openSl) to ++(0,\blndEx);
  \node[anchor=south east] at (\sliDis-0.1,\slitMi+0.1) {$1$};
  \node[anchor=north east] at (\sliDis-0.1,-\slitMi-0.1) {$2$};
  \draw (0,0) to ++(\sliDis,-\slitMi);
  \foreach \x in {-\scdArr,...,\scdArr} \draw[->] (\sliDis,-\slitMi) to ++(\x*\scdAng/\wavArr:\scArLn);
  \foreach \x in {1,...,\scdWav} \draw (\sliDis,-\slitMi) ++(0,-\x*\scdFDs) arc(-90:90:\x*\scdFDs);
  \draw (0,0) to ++(\sliDis,\slitEx+\openSl/2);
  \foreach \x in {-\scdArr,...,\scdArr} \draw[->] (\sliDis,\slitEx+\openSl/2) to ++(\x*\scdAng/\wavArr:\scArLn);
  \foreach \x in {1,...,\scdWav} \draw (\sliDis,\slitEx+\openSl/2) ++(0,-\x*\scdFDs) arc(-90:90:\x*\scdFDs);

  \draw[fill=white!80!black,shift={(\scrnDs,0)},samples=100,domain=-\scrnEx:\scrnEx] (0,-\scrnEx) plot ({-\pltIth*((sin(360*\lamDee*sin(atan(\pltStr*\x))))/(sin(atan(\pltStr*\x))))^2},\x) -- (0,\scrnEx) -- cycle;

\end{tikzpicture}

\end{table}

Die Trajektorien $\norBra{\vX\of{t},\vP\of{t}}$ der Teilchen gehen als Ladungs- und Stromdichten in die inhomogenen Maxwellgleichungen ein; Die Lösungen $\vB\of{\vX,t}$, $\vB\of{\vX,t}$ der Maxwellgleichungen gehen über die Lorentzkraft in die Bewegungsgleichungen der Teilchen ein.

\subsection{Empirische Grundlagen der Quantenmechanik}

\subsubsection{Wellen haben Teilchencharakter}

\begin{itemize}
  \item Erste Quantenhypothese durch Max Planck (1900) zur Erklärung des Spektrums der \Hl{Schwarz\-kör\-per\-strahl\-ung}: die Energie einer Welle der (Kreis-)Frequenz $\omega = 2\pi\nu$ ist ein ganzzahliges Vielfaches eines elementaren Energiequantums
  \begin{equation}
    E = h \nu = \hbar \omega \quad \hbar = \frac{h}{2\pi}
  \end{equation}
  Das \Hl{Plancksche Wirkungsquantum} $h$ hat die Einheit einer $\text{Wirkung} = \text{Energie} \times \text{Zeit}$, $\Unit{h} = \textbf{Js}$. Aus Plancks Quantenhypothese ergibt sich das \Hl{Plancksche Strahlungsgesetz}. Es beschreibt das empirisch bekannte Spektrum thermischer Strahlung, wenn
  \begin{equation}
    h = 6.62\cdot 10^{-34} \textbf{Js} \quad \hbar = 1.05\cdot 10^{-34} \textbf{Js}
  \end{equation}
  gewählt wird.
  \item Albert Einsteins Erklärung für den \Hl{photoelektrischen Effekt} (1905): Licht der Frequenz $\omega$ besteht aus Teilchen der Energie $E = \hbar \omega$ und die Lichtteilchen (Photonen) haben einen Impuls
  \begin{equation}
    \vP = \hbar \vK \quad \text{($\vK$ Wellenvektor)}
  \end{equation}
  \item \Hl{Comptoneffekt} (1924, Vergrößerung der Wellenlänge des Lichts bei Streuung an Elektronen) kann mit $E = \hbar \omega$, $\vP = \hbar \vK$ als elastischer Stoß zweier Teilchen erklärt werden.
  \item Doppelspaltexperiment mit Einzelphotonauflösung (Abb. nach Schwabl, Quantenmechanik)
\end{itemize}

\begin{figure}[H]
  \centering
  \tikzset{thick,every node/.style={scale=1}}
\begin{tikzpicture}
  \def\wavFrt{3};
  \def\wavArr{3};
  \def\arrAng{60};
  \def\arrLen{1.7};
  \def\frtDis{0.7};
  \def\sliDis{2.5};
  \def\slitEx{0.8};
  \def\openSl{0.3};
  \def\blndEx{1.4};
  \def\scrnDs{4.5};
  \def\scdWav{2};
  \def\scdFDs{0.3};
  \def\scdArr{2};
  \def\scdAng{100};
  \def\scArLn{0.8};
  \def\lensLn{0.5};
  \def\lenThk{0.1};
  \def\imgDis{6};
  \def\pltThk{0.05};
  \def\pltStr{0.8};
  \def\pltIth{0.01};
  \def\lamDee{2.1};
  \tikzmath{\scrnEx = \slitEx+\openSl+\blndEx;
  \slitMi = \slitEx+\openSl/2;}

  \fill (0,0) circle(0.1);
  \foreach \x in {1,...,\wavFrt} \draw (0,-\x*\frtDis) arc(-90:90:\x*\frtDis);
  \foreach \x in {-\wavArr,...,\wavArr} \draw[->] (0,0) to ++(\x*\arrAng/\wavArr:\arrLen);
  \node[anchor=south] at (\sliDis,\scrnEx+0.1) {Doppelspalt};
  \node[anchor=south] at (\scrnDs,\scrnEx+0.1) {Schirm};
  \node[anchor=north] at (\scrnDs,-\scrnEx-0.1) {Teilchendichte $\rho\of{\vX}$};
  \node[anchor=south west] at (0,\scrnEx+0.1) {Quelle};
  \draw (\sliDis,-\slitEx-\openSl-\blndEx) to ++(0,\blndEx) ++(0,\openSl) to ++(0,2*\slitEx) ++(0,\openSl) to ++(0,\blndEx);
  \node[anchor=south east] at (\sliDis-0.1,\slitMi+0.1) {$1$};
  \node[anchor=north east] at (\sliDis-0.1,-\slitMi-0.1) {$2$};
  \draw (0,0) to ++(\sliDis,-\slitMi);
  \foreach \x in {-\scdArr,...,\scdArr} \draw[->] (\sliDis,-\slitMi) to ++(\x*\scdAng/\wavArr:\scArLn);
  \foreach \x in {1,...,\scdWav} \draw (\sliDis,-\slitMi) ++(0,-\x*\scdFDs) arc(-90:90:\x*\scdFDs);
  \draw (0,0) to ++(\sliDis,\slitEx+\openSl/2);
  \foreach \x in {-\scdArr,...,\scdArr} \draw[->] (\sliDis,\slitEx+\openSl/2) to ++(\x*\scdAng/\wavArr:\scArLn);
  \foreach \x in {1,...,\scdWav} \draw (\sliDis,\slitEx+\openSl/2) ++(0,-\x*\scdFDs) arc(-90:90:\x*\scdFDs);

  \draw[fill=white!80!black,shift={(\scrnDs,0)},samples=100,domain=-\scrnEx:\scrnEx] (0,-\scrnEx) plot ({-\pltIth*((sin(360*\lamDee*sin(atan(\pltStr*\x))))/(sin(atan(\pltStr*\x))))^2},\x) -- (0,\scrnEx) -- cycle;

  \draw[fill=white!80!black,shift={(\imgDis+\scrnDs,0)},samples=100,domain=-\scrnEx:\scrnEx] (0,-\scrnEx) plot ({-\pltIth*((sin(360*\lamDee*sin(atan(\pltStr*\x))))/(sin(atan(\pltStr*\x))))^2},\x) -- (0,\scrnEx) -- cycle;

  \node[anchor=north] at (\scrnDs+\imgDis,-\scrnEx-0.1) {Schirmintensität $\rho\of{\vX} = \Norm{\psi\of{\vX}}^2$};

\end{tikzpicture}

\end{figure}

\subsubsection{Teilchen haben Wellencharakter}

\begin{itemize}
  \item \Hl{Bohrsches Atommodell} (1913)
  \begin{enumerate}
    \item Elektronen bewegen sich auf Kreisbahnen. Sie strahlen dabei keine elektromagnetische Energie ab.
    \item Erlaubt sind nur Kreisbahnen mit Drehimpuls
    \begin{equation}
      L = n \hbar \quad n = 1,2,3
    \end{equation}
    \item Elektromagnetische Energie wird nur bei einem Übergang zwischen zwei Kreisbahnen abgestrahlt.
  \end{enumerate}
  \Hl{Konsequenzen aus den Bohrschen Postulaten}:
  \item Mit dem Virialsatz
  \begin{equation}
    E_\text{kin} = - \frac{1}{2} E_\text{pot} \quad \frac{m v^2}{2} = \frac12 \frac{e^2}{4\pi\epsilon_0}\frac{1}{r}
  \end{equation}
  folgt aus der Forderung $L = m r v = n \hbar$ die Geschwindigkeit und der Radius für die $n$-te Kreisbahn
  \begin{equation}
    v_n = \alpha c \frac{1}{n} \quad r_n = a_0 n^2
  \end{equation}
  mit der Feinstrukturkonstanten $\alpha$ und dem Bohrradius $a_0$
  \begin{equation}
    \alpha = \frac{e^2}{4\pi\epsilon_0}\frac{1}{\hbar c} \simeq \frac{1}{137} \quad a_0 = \frac{\hbar}{\alpha m c} \simeq 0.5\cdot 10^{-10} \textbf{m}
  \end{equation}
\end{itemize}
\begin{figure}[H]
  \centering
  \tikzset{thick,every node/.style={scale=1}}
\begin{tikzpicture}
  \def\innCir{0.3};
  \def\outCir{3};
  \def\midCir{1};
  \def\innAng{-10};
  \def\midAng{20};
  \def\outAng{40};
  \def\lenVel{1};

  \draw[red] (0,0) circle(\innCir);
  \draw[orange] (0,0) circle(\midCir);
  \draw[green] (0,0) circle(\outCir);
  \draw[->] (0,0) to node[pos=2] {$r_1$} (\innAng:\innCir);
  \draw[->] (0,0) to node[pos=1.3] {$r_2$} (\midAng:\midCir);
  \draw[->] (0,0) to node[pos=1.1] {$r_3$} (\outAng:\outCir);
  \draw[->,red!50!orange] (\outAng:\outCir) to node[above] {$v_3$} ++(\outAng+90:\lenVel);
\end{tikzpicture}

\end{figure}
\begin{itemize}
  \item Die Energie der $n$-ten Kreisbahn ist
  \begin{equation}
    E_n = - \frac12 \frac{m\alpha^2c^2}{n^2}
  \end{equation}
  Bei einem Übergang von der $n$-ten zur $m$-ten Kreisbahn wird Strahlung emittiert mit einer Frequenz
  \begin{equation}
    \nu = \frac{E_n-E_m}{h} = R \norBra{\frac{1}{m^2} - \frac{1}{n^2}}
  \end{equation}
  Wobei $R = \frac{m\alpha^2c^2}{2 h}$ die Rydbergkonstante ist.
  \item Luis de Broglie(1924): Die Bewegung eines massiven Teilchens mit Impuls $p$ entspricht einer \glqq Materiewelle\grqq\ mit einer \Hl{de-Broglie-Wellenlänge}
  \begin{equation}
    \lambda = \frac{h}{p}
  \end{equation}
  Damit entsprechen die Bohrschen Kreisbahnen stehenden Wellen der Materiewelle
\end{itemize}
\begin{figure}[H]
  \centering
  \tikzset{thick,every node/.style={scale=1}}
\begin{tikzpicture}
  \def\nucRad{0.3};
  \def\cirRad{3};
  \def\sinAmp{1};
  \def\sinPer{2};

  \fill (0,0) circle(\nucRad);
  \node at (0,2*\nucRad) {Nukleus};
  \draw (0,0) circle(\cirRad);
  \draw[domain=0:360,samples=200] plot (\x:{\cirRad-\sinAmp/2+\sinAmp*(sin((\x-22.5)*\sinPer))^2});
  \draw[domain=0:360,samples=200] plot (\x:{\cirRad-\sinAmp/2+\sinAmp*(sin((\x+22.5)*\sinPer))^2});
  \draw (0,0) to node[pos=0.5,fill=white] {$r$} (\cirRad,0);

\end{tikzpicture}

\end{figure}

\begin{equation}
  \lambda_n = \frac{h}{m v_n} = 2\pi a_0 n \quad \rightarrow \quad n \lambda_n = 2\pi a_0 n^2 = 2 \pi r_n
\end{equation}

\begin{itemize}
  \item Direkter Nachweis der Wellennatur geschieht in Interferenzexperimenten:\\
  Davisson und Germer (1927): Elektron an Kristallgittern\\
  Jönssen (1954): Doppelspaltexperiment mit Elektronen\\
  Inerferometrie mit organischen Molekülen
\end{itemize}

\subsection{Die Wellenfunktion}
[Es folgt eine provisorische und vereinfachte Darstellung der Postulate der nichtrelativistischen Quantenmechanik. Die exakte und vollständige Darstellung folgt später in Kapitel 2.]
\begin{itemize}
  \item Die Erfahrung zeigt:
  \begin{itemize}
    \item Der \Hl{Zustand} eines Teilchens wird durch eine (komplexe) \Hl{Wellenfunktion}
    \begin{equation}
      \psi\of{\vX,t}
    \end{equation}
    beschrieben.
    \item Seine \Hl{Dynamik} (zeitliche Entwicklung, Evolution) ist durch die \Hl{Schrödingergleichung} festgelegt
    \begin{empheq}[box=\widefbox]{equation}
      \mI \hbar \ppv{\psi\of{\vX,t}}{t} = - \frac{\hbar^2}{2 m} \Laplace \psi\of{\vX,t} + V\of{\vX,t}\psi\of{\vX,t}
    \end{empheq}
    Die Lösung der Schrödingergleichung unter einer Anfangsbedingung $\psi\of{\vX,0}$ ergibt die zeitlich entwickelte Wellenfunktion $\psi\of{\vX,t}$.
    \item Die Wellenfunktion $\psi\of{\vX,t}$ legt die \Hl{Statistik} der Resultate von Messungen physikalischer Messgrößen $A\of{\vX,\vP}$ zum Zeitpunkt $t$ fest:\\
    Für die Messung der \Hl{Position} des Teilchens gilt (Bornsche Regel): Die Wahrscheinlichkeit, das Teilchen zum Zeitpunkt $t$ in einem Volumen $\dn{x}{3}$ vorzufinden,ist
    \begin{equation}
      P\of{\vX,\vP} = \Norm{\psi\of{\vX,t}}^2 \dn{x}{3}
    \end{equation}
    Das heißt, die \Hl{Wahrscheinlichkeitsdichte} im Raum ist
    \begin{equation}
      \rho\of{\vX,t} = \Norm{\psi\of{\vX,t}}^2 \quad \norBra{\Unit{\rho\of{\vX,t}} = \mtr^{-3}}
    \end{equation}
  \end{itemize}
  \item Die Wahrscheinlichkeit, das Teilchen irgendwo im Raum vorzufinden, muss $1$ sein:
  \begin{equation}
    1 = \int \dn{x}{3} \rho\of{\vX,t} = \int \dn{x}{3} \Norm{\psi\of{\vX,t}}^2
  \end{equation}
  Wir fordern daher, dass eine als Wellenfunktion in Frage kommende Funktion $\psi\of{\vX,t}$
  \begin{itemize}
    \item quadratintegrabel und
    \item auf $1$ normiert ist.
  \end{itemize}
  Eine Wellenfunktion (für den Bewegungszustand eines Teilchens) ist damit Element des \Hl{Hilbert\-raumes} $L^2\of{\mathds{R}^3}$ der quadratintegrablen Funktionen,
  \begin{equation}
    L^2\of{\mathds{R}^3} = \setProp{\psi : \mathds{R}^3 \to \mathds{C}}{\int \dn{x}{3} \Norm{\psi\of{\vX}}^2 < \infty}
  \end{equation}
  \item Die \Hl{Schrödingergleichung für ein freies Teilchen} ($V\of{\vX,t} \equiv 0$)
  \begin{equation}
    \mI \ppv{}{t}\psi\of{\vX,t} = - \frac{\hbar^2}{2 m} \Laplace \psi\of{\vX,t}
  \end{equation}
  wird durch \Hl{ebene Wellen} gelöst
  \begin{equation}
    \psi\of{\vX,t} = c \e{\mI\norBra{\vK\cdot\vX - \omega t}}
  \end{equation}
  mit $c \in \mathds{C}$ und der Dispersionsrelation
  \begin{equation}
    \omega = \frac{\hbar \vK^2}{2 m}
  \end{equation}
  Mit $E = \hbar \omega$ und $\vP = \hbar \vK$ (beziehungsweise $p = \frac{\hbar 2 \pi}{\lambda} = \frac{h}{\lambda}$) ist das äquivalent zu $E = \frac{\vP^2}{2 m}$. Eine ebene Materiewelle mit Wellenvektor $\vK$ beschreibt die Bewegung eines freien Teilchens mit Energie
  \begin{equation}
    E = \frac{\norBra{\hbar\vK}^2}{2 m}
  \end{equation}
  Die Wahrscheinlichkeitsdichte einer ebenen Welle ist
  \begin{equation}
    \rho\of{\vX,t} = \Norm{\psi\of{\vX,t}}^2 = \Norm{c}^2 \equiv \text{konstant}
  \end{equation}
  Die Aufenthaltswahrscheinlichkeit ist also gleichmäßig im ganzen Raum. Ebene Wellen sind nicht normierbar (nicht Element von $L^2\of{\mathds{R}^3}$)! Sie sind keine physikalischen Zustände.\\
  Durch Superposition von ebenen Wellen können normierbare, physikalische Zustände aufgebaut werden, sogenannte \Hl{Wellenpakete}:
  \begin{equation}
    \psi\of{\vX,t} = \frac{1}{\norBra{2\pi}^\frac{3}{2}}\int \dn{k}{3} c\of{\vK} \e{\mI\edgBra{\vK\cdot\vX-\omega\of{\vK} t}}
  \end{equation}
  löst die Schrödingergleichung, wenn $\omega\of{\vK} = \frac{\hbar\vK^2}{2m}$ für beliebige Amplituden $c\of{\vK}$.\\
  Normierung erfordert
  \begin{equation}
    1 = \int \dn{x}{3} \Norm{\psi\of{\vX,t}}^2= \int \dn{k}{3} \Norm{c\of{\vK}}^2
  \end{equation}
  Das heißt, es muss $c\of{\vK}$ selbst quadratintegrabel und normiert sein.
  \item Die lösung der Schrödingergleichung für eine vorgegebene Anfangsbedingung $\psi\of{\vX,0}$ zum Zeitpunkt $t = 0$ erhält man aus der Bedingung
  \begin{equation}
    \psi\of{\vX,0} = \frac{1}{\norBra{2\pi}^\frac{3}{2}}\int \dn{k}{3} c\of{\vK} \e{\mI\vK\cdot\vX}
  \end{equation}
  Also ist $\psi\of{\vX,0}$ die Fouriertransformierte von $c\of{\vK}$. Die inverse Transformation ergibt
  \begin{equation}
    c\of{\vK} = \frac{1}{\norBra{2\pi}^{\frac32}} \int \dn{x}{3} \psi\of{\vX,0} e^{-\mI\vK\cdot\vX}
  \end{equation}
  Einsetzen der aus der Anfangsbedingung $\psi\of{\vX,0}$ bestimmten Amplituden in die Allgemeine Lösung ergibt die Wellenfunktion zu Zeiten $t$
  \begin{equation}
    \psi\of{\vX,t} = \frac{1}{\norBra{2\pi}^\frac{3}{2}} \int \dn{k}{3} c\of{\vK} \e{\mI\edgBra{\vK\cdot\vX - \omega t}}
  \end{equation}
\end{itemize}

\subsection{Die Impulswellenfunktion}
\begin{itemize}
  \item Die Koeffizienten $c\of{\vK}$ sind die Amplituden der ebenen Wellen $\e{\mI\vK\cdot\vX}$ (mit festem Impuls $\vP = \hbar \vK$) im Wellenpaket $\psi\of{\vX}$.\\
  Wir erwarten, dass $\Norm{c\of{\vK}}^2$ die Wahrscheinlichkeit festlegt, bei einer Messung des Impulses $\vP$ an einem Teilchen im Zustand $\psi\of{\vX}$ den Wert $\vP = \hbar\vK$ zu finden.
  \item Allgemein definieren wir die \Hl{Impulswellenfunktion} $\varphi\of{\vP,t}$ bei gegebener \Hl{(Raum)-Wellenfunktion} $\psi\of{\vX,t}$
  \begin{equation}
    \varphi\of{\vP,t} = \frac{1}{\norBra{2\pi}^\frac{3}{2}} \int \dn{x}{3} \psi\of{\vX,t} \e{-\mI\frac{\vP\cdot\vX}{\hbar}}
  \end{equation}
  (Damit gilt $\varphi\of{\vP,0} = \hbar^{-\frac{3}{2}}c\of{\frac{\vP}{\hbar}}$.)\\
  Die Wahrscheinlichkeit, bei einer Messung des Impulses einen Wert $\vP$ zu finden ist
  \begin{equation}
    \prob{\vP,t} = \Norm{\varphi\of{\vP}}^2\dn{p}{3}
  \end{equation}
  Die Wahrscheinlichkeitsdichte im Impulsraum ist
  \begin{equation}
    \Norm{\varphi\of{\vP}}^2
  \end{equation}
  Es gilt
  \begin{equation}
    \int \dn{p}{3} \Norm{\varphi\of{\vP}}^2 = 1
  \end{equation}
\end{itemize}

\subsection{Wellenmechanik in einer Dimension}
\begin{itemize}
  \item Wir betrachten im Folgenden die Bewegung eines Teilchens in \Hl{einer Dimension}. Sein Zustand wird beschrieben durch eine Wellenfunktion $\varphi\of{x,t}$ in
  \begin{equation}
    L^2\of{\mathds{R}} = \setProp{\varphi:\mathds{R} \to \mathds{C}}{\int\dd{x}\Norm{\varphi\of{x}}^2< \infty}
  \end{equation}
  \item Ein Teilchen mit Energie $E = \frac{\norBra{\hbar k}^2}{2 m}$ wird beschrieben durch eine \Hl{ebene Welle}
  \begin{equation}
    \varphi\of{x,t} = \e{\mI\norBra{kx - \omega t}}
  \end{equation}
  \item Ein \Hl{Wellenpaket} in einer Dimension ist
  \begin{equation}
    \varphi\of{x,t} = \frac{1}{\sqrt{2\pi}} \int\limits_{-\infty}^{\infty} \dd{k} c\of{k} \e{\mI\norBra{kx-\omega t}}
  \end{equation}
  mit $\omega = \frac{\hbar k^2}{2 m}$ und $\int\limits_{-\infty}^{\infty} \dd{k} \Norm{c\of{k}}^2 = 1$. Die zugehörige Impulswellenfunktion ist
  \begin{equation}
    \psi\of{p,t} = \frac{1}{\sqrt{2\pi \hbar}} \int\limits_{-\infty}^{\infty} \dd{x} \varphi\of{x,t} \e{-\mI\frac{x}{\hbar}}
  \end{equation}
  Insbesondere für $t = 0$ ist $\psi\of{p,0} = \frac{1}{\hbar} c \of{\frac{p}{\hbar}}$
  \item Beispiel: \Hl{Gaußsches Wellenpaket}
  \begin{equation}
    \varphi\of{x,0} = \norBra{\frac{2}{\pi a^2}}^\frac{1}{4} \e{\mI k_0 x - \frac{x^2}{a^2}}
  \end{equation}
  mit $k_0,a \in \mathds{R}$. Die  Wahrscheinlichkeitsdichte im Ort ist
  \begin{equation}
    \rho\of{x,0} = \Norm{\varphi\of{x,0}}^2 = \norBra{\frac{2}{\pi a^2}}^\frac{1}{2} \e{-2\frac{x^2}{a^2}}
  \end{equation}
  Es gilt:
  \begin{align}
    & \int \dd{x} \rho\of{x} = 1 \\
    \erw{x} = & \int \dd{x} x \rho\of{x} = 0 \\
    \erw{x^2} = & \int \dd{x} x^2 \rho\of{x} = \frac{a^2}{4} \\
    \Delta x = & \sqrt{\erw{x^2}-\erw{x}^2} = \frac{a}{2}
  \end{align}
\end{itemize}
\begin{figure}[H]
  \centering
  \tikzset{thick,every node/.style={scale=1}}
\begin{tikzpicture}
  \def\a{1.5};

  \draw[->] (-5.2,0) to (5.2,0);
  \draw[->] (0,-0.15) to (0,5.2);
  \node[anchor=west] at (5.3,0) {$x$};
  \node[anchor=south] at (0,5.3) {$\Norm{\psi\of{x}}^2$};
	\node[anchor=north] at (0,-0.25) {$x = 0$};
  \draw[thick,domain=-5.2:5.2,samples=100] plot (\x,{5*exp(-2*\x*\x/\a/\a)});
  \draw[<->] (-\a/2,{5*exp(-1/2)-0.2}) to node[anchor=north west] {$\frac{a}{2}$}++ (\a,0);
\end{tikzpicture}

\end{figure}
\begin{itemize}
  \item Die Amplituden $c\of{k}$ sind
  \begin{align}
    c\of{k} & = \frac{1}{\sqrt{2\pi}} \int\limits_{-\infty}^{\infty} \dd{x} \varphi\of{x,0}\e{-\mI k x} \\
    & = \norBra{\frac{a^2}{2\pi}}^\frac{1}{4} \e{-\norBra{k-k_0}^2\frac{a^2}{4}}
  \end{align}
  \item Die Impulswellenfunktion zum Zeitpunkt $t = 0$ ist
  \begin{equation}
    \psi\of{p,0} = \norBra{\frac{a^2}{2\pi \hbar^2}}^\frac{1}{4} \e{-\norBra{p-p_0}^2\frac{a^2}{4\hbar^2}}
  \end{equation}
	Es gilt:
	\begin{align}
		& \int \dd{p} \Norm{\psi\of{p,0}}^2 = 1 \\
		\erw{p} = & \int \dd{p} p \Norm{\psi\of{p,0}}^2 = p_0 = \hbar k_0 \\
		\erw{p^2} = & \int \dd{p} p^2 \Norm{\psi\of{p,0}}^2 = p_0^2 + \frac{\hbar^2}{a^2} \\
		\Delta p = & \sqrt{\erw{p^2}- \erw{p}^2} = \frac{\hbar}{a}
	\end{align}
\end{itemize}
\begin{figure}[H]
	\centering
	\tikzset{thick,every node/.style={scale=1}}
\begin{tikzpicture}
  \def\a{1.5};
	\def\po{7};

  \draw[->] (-0.15,0) to (10.4,0);
  \draw[->] (0,-0.15) to (0,5.2);
  \node[anchor=west] at (10.5,0) {$p$};
  \node[anchor=south] at (0,5.3) {$\Norm{\varphi\of{p}}^2$};
  \draw[thick,domain=0:10.4,samples=100] plot (\x,{5*exp(-2*(\x-\po)*(\x-\po)/\a/\a)});
  \draw[<->] (\po-\a/2,{5*exp(-1/2)-0.2}) to  node[anchor=north west] {$\frac{\hbar}{2}$} ++ (\a,0);
	\draw[dashed] (\po,0) to (\po,5);
	\draw (\po,-0.15) to ++(0,0.3);
	\node[anchor=north] at (\po,-0.25) {$p_0$};
\end{tikzpicture}

\end{figure}
Das heißt, dass die Unschärfen in Ort und Impuls nicht unabhängig sind
\begin{equation}
	\Delta x \Delta p = \frac{\hbar}{2}
\end{equation}
\begin{itemize}
	\item Das zeitlich entwickelte Wellenpaket
	\begin{equation}
		\psi\of{x,t} = \frac{11}{\sqrt{2\pi}}\int\limits_{-\infty}^{\infty} \dd{k} c\of{k} \e{\mI\norBra{kx - \frac{\hbar k^2}{2 m} t}} = \norBra{\frac{2}{\pi a^2}}^\frac14 \frac{\e{\mI\norBra{k_0 x - \phi}}}{\norBra{1 + \gamma^2 t^2}^{1/4}} \e{-\frac{x-v_0 t}{a^2\norBra{1 + \mI \gamma t}}}
	\end{equation}
	wobei
	\begin{equation}
		v_0 = \frac{\hbar k_0}{m} = \frac{p_0}{m} \quad \gamma = \frac{2\hbar}{m a^2} \quad \varphi = \theta + \frac{k_0 v_0 t}{2} \quad \tan\of{2\theta} = \gamma t
	\end{equation}
	Die Wahrscheinlichkeitsdichte ist
	\begin{align}
		\rho\of{x,t} & = \Norm{\psi\of{x,t}}^2 \\
		 & = \norBra{\frac{2}{\pi a^2}}^\frac12\frac{1}{\norBra{1 + \gamma^2 t^2}^{1/2}}\e{-\frac{2\norBra{x-x_0}^2}{a^2\norBra{1 + \gamma^2 t^2}}}
	\end{align}
	mit dem Mittelwert
	\begin{align}
		\erw{x}_t & = v_0 t \\
		\Delta x_t & = \frac{a}{2}\sqrt{1 +\gamma^2 t^2}
	\end{align}
\end{itemize}
\begin{figure}
	\centering
	\tikzset{thick,every node/.style={scale=1}}
\begin{tikzpicture}
  \def\a{0.5};
	\def\at{1.1};
	\def\xo{6};

  \draw[->] (-2.2,0) to (8.2,0);
  \draw[->] (0,-0.15) to (0,5.2);
  \node[anchor=west] at (8.3,0) {$x$};
  \node[anchor=south] at (0,5.3) {$\Norm{\psi\of{x}}^2$};
	\node[anchor=north] at (0,-0.25) {$x = 0$};
	\draw[thick,domain=-2.2:8.2,samples=200] plot (\x,{5*exp(-2*\x*\x/\a/\a)});
  \draw[<->] (-\a/2,{5*exp(-1/2)-0.2}) to node[anchor=west,pos=1.1] {$\frac{a}{2}$}++ (\a,0);
  \draw[thick,domain=-2.2:8.2,samples=100] plot (\x,{\a/\at*5*exp(-2*(\x-\xo)*(\x-\xo)/\at/\at)});
  \draw[<->] (\xo-\at/2,{\a/\at*5*exp(-1/2)-0.2}) to  node[anchor=east,pos=-0.1] {$\frac{a}{2}\sqrt{1+\gamma^2t^2}$} ++ (\at,0);
	\draw[dashed] (\xo,0) to node[above,pos=1.1] {$t>0$} (\xo,\a/\at*5);
	\draw (\xo,-0.15) to ++(0,0.3);
	\node[anchor=north] at (\xo,-0.25) {$v_0 t$};
\end{tikzpicture}

\end{figure}
\begin{itemize}
	\item Es gibt keine Änderungen des Impulses beziehungsweise der Statistik von Impulsmessungen für ein freies Teilchen und
	\begin{equation}
		\Norm{\varphi\of{p,t}}^2 = \Norm{\varphi\of{p,0}}^2
	\end{equation}
\end{itemize}

\subsection{Erhaltung der Wahrscheinlichkeit}
\begin{itemize}
	\item In der Elektrodynamik gilt für die Ladungsdichte $\rho_q\of{\vX,t}$ und die Stromdichte $\vJ_q\of{\vX,t}$ der Erhaltungssatz ($\rightarrow$ Kontinuitätsgleichung)
	\begin{equation}
		\ppv{}{t} \rho_q\of{\vX,t} + \Nabla \cdot \vJ_q\of{\vX,t} = 0
	\end{equation}
	Die Gesamtladung ist erhalten
	\begin{align}
		\ppv{}{t}\int\nolimits_V \dn{x}{3} \rho_q\of{\vX,t} & = - \int\nolimits_V \dn{x}{3} \Nabla\cdot\vJ_q\of{\vX,t} \\
		& = - \int\nolimits_{\partial V} \dd{\vS}\cdot\vJ_q\of{\vX,t} \\
		& = 0 \text{, für } V\to\infty \text{ wegen } \lim\limits_{x\to \infty} \vJ_q\of{\vX,t} = 0
	\end{align}
	\item In der Quantenmechanik gilt ein entsprechender \Hl{Erhaltungssatz der Wahrscheinlichkeit}. Wir betrachten die Änderung der Wahrscheinlichkeitsdichte
	\begin{align}
		\ppv{}{t} \rho\of{\vX,t} & = \ppv{}{t} \Norm{\psi\of{\vX,t}}^2 = \ppv{}{t} \psi^*\of{\vX,t}\psi\of{\vX,t} \\
		& = \psi\of{\vX,t}\dot{\psi}^*\of{\vX,t} + \psi^*\of{\vX,t}\dot{\psi}\of{\vX,t}
	\end{align}
	Die Schrödingergleichung
	\begin{equation}
		\mI \hbar \dot{\psi} = - \frac{\hbar^2}{2 m} \Laplace \psi + V\psi
	\end{equation}
	impliziert
	\begin{equation}
		-\mI\hbar\dot{\psi}^* = - \frac{\hbar^2}{2m}\Laplace\psi^* + V\psi^*
	\end{equation}
	Damit ergibt sich
	\begin{equation}
		\psi^*\dot{\psi} + \psi\dot{\psi}^* = \frac{1}{\mI\hbar}\norBra{-\frac{\hbar^2}{2m}\psi^*\Laplace\psi + V \psi^*\psi} - \frac{1}{\mI\hbar}\norBra{-\frac{\hbar^2}{2 m} \psi\Laplace\psi^* + V\psi^*\psi}
	\end{equation}
	Also
	\begin{equation}
		\ppv{}{t} \rho\of{\vX,t} + \frac{\hbar}{2m\mI}\norBra{\psi^*\Laplace\psi - \psi\Laplace\psi^*} = 0
	\end{equation}
	Wir definieren den \Hl{Wahrscheinlichkeitsstrom}
  \begin{empheq}[box=\widefbox]{align}
		\vJ\of{\vX,t} & = \frac{\hbar}{2 m \mI}\norBra{\psi^*\of{\vX,t} \Nabla\psi\of{\vX,t} - \psi\of{\vX,t}\Nabla\psi^*\of{\vX,t}} \\
		& = \frac{1}{m} \Re \edgBra{\psi^*\of{\vX,t}\norBra{-\mI\hbar\Nabla}\psi\of{\vX,t}}
	\end{empheq}
	so, dass
	\begin{equation}
		\Nabla\cdot\vJ = \frac{\hbar}{2 m \mI} \norBra{\Nabla\psi^*\cdot\Nabla\psi + \psi^*\Laplace\psi - \Nabla\psi\cdot\Nabla\psi^* - \psi\Laplace\psi^*}
	\end{equation}
	Damit gilt der \Hl{Erhaltungssatz der Wahrscheinlichkeit}
  \begin{empheq}[box=\widefbox]{equation}
		\ppv{}{t}\rho\of{\vX,t} + \Nabla\cdot\vJ = 0
	\end{empheq}
	Die Gesamtwahrscheinlichkeit $\int\dn{x}{3}\rho\of{\vX,t} = 1$ ist erhalten (Beweis wie in der Elektrodynamik; $\lim\limits_{x\to\infty}\vJ\of{\vX,t} = 0$ weil $\lim\limits_{x\to\infty} \psi\of{\vX,t} = 0$)
	\item Beispiel: Der Wahrscheinlichkeitsstrom für eine ebene Welle $\psi\of{\vX,t} = \e{\mI\norBra{\vK\vX - \omega t}}$ ist
	\begin{align}
		\vJ\of{\vX,t} & = \frac{\hbar}{2m\mI} \norBra{\psi^*\norBra{\mI\vK}\psi - \psi\norBra{-\mI\vK}\psi^*} \\
		& = \frac{2 \mI\hbar \vK}{2 m\mI} \psi^*\psi \\
		& = \frac{\hbar\vK}{m} = \frac{\vP}{m} = \vV
	\end{align}
	\item Beispiel: Superpostition zweier ebener Wellen in $1$D
	\begin{equation}
		\psi\of{x,t} = A\e{\mI\norBra{kx-\omega t}} + B\e{\mI\norBra{-kx-\omega t}}
	\end{equation}
	der Strom ist
	\begin{equation}
		\vJ = \frac{\hbar\vK}{m}\norBra{\Norm{A}^2-\Norm{B}^2}\vUn_x
	\end{equation}
\end{itemize}

\subsection{Zeitunabhängige Schrödingergleichung}

\begin{itemize}
	\item Falls das Potential $V\of{\vX}$ nicht explizit von der Zeit abhängt, so ist die Schrödingergleichung
	\begin{align}
		\mI\hbar\ppv{}{t}\Psi\of{\vX,t} & = \norBra{-\frac{\hbar^2}{2m}\Laplace + V\of{\vX}}\Psi\of{\vX,t} \\
		& = H\of{\vX}\Psi\of{\vX,t}
	\end{align}
	$H\of{\vX}$ wird als \Hl{Hamiltonoperator} des Systems bezeichnet.
	\item \Hl{Separation der Variablen}: Wir suchen Lösungen der Art
	\begin{equation}
		\Psi\of{\vX,t} = \chi\of{t}\psi\of{\vX}
	\end{equation}
	Damit folgt:
	\begin{equation}
		\mI\hbar\dot{\chi}\of{t}\psi\of{\vX} = \chi\of{t}H\of{\vX}\psi\of{\vX} \quad \rightarrow \quad \mI\frac{\dot{\chi}\of{t}}{\chi\of{t}} = \frac{H\of{\vX}\psi\of{\vX}}{\psi\of{\vX}} \equiv E
	\end{equation}
	mit einer Separationskonstanten $E$ (der Dimension Energie). Also fordern wir
	\begin{align}
		\mI\hbar\frac{\dot{\chi}\of{t}}{\chi\of{t}} = E \label{sep1} \\
		\frac{H\psi\of{\vX}}{\psi\of{\vX}} = E \label{sep2}
	\end{align}
	\item \eqref{sep1} äquivalent zu
	\begin{equation}
		\dot{\chi}\of{t} = - \mI \frac{E}{\hbar}\chi\of{t} \quad \rightarrow \quad \chi\of{t} = \e{-\mI\frac{E t}{\hbar}}\chi\of{0} = \e{-\mI\frac{E t}{\hbar}}
	\end{equation}
	\item \eqref{sep2} ergibt die \Hl{zeitunabhängige Schrödingergleichung}
  \begin{empheq}[box=\widefbox]{align}
		H\of{\vX}\psi\of{\vX} & = E \psi\of{\vX} \\
		\norBra{-\frac{\hbar^2}{2m}\Laplace + V\of{\vX}}\psi\of{\vX} & = E \psi\of{\vX}
	\end{empheq}
	Dies ist eine \Hl{Eigenwertgleichung}:
	\begin{itemize}
		\item $\psi\of{\vX}$ ist eine \Hl{Eigenfunktion} (\Hl{Energieeigenzustand} oder \Hl{stationärer Zustand}) des Hamiltonoperators $H$.
		\item $E$ ist der zur Eigenfunktion $\psi\of{\vX}$ gehörende \Hl{Eigenwert} (\Hl{Energieeigenwert}).
	\end{itemize}
	Für eine gegebene Eigenfunktion $\psi\of{\vX}$ mit Eigenwert $E$ ist die Lösung der zeitabhängigen Schrö\-dinger\-gleichung
	\begin{equation}
		\Psi\of{\vX,t} = \psi\of{\vX}\e{\mI\frac{E t}{\hbar}}
	\end{equation}
	\item Im Allgemeinen besitzt ein Hamiltonoperator mehrere (unendlich viele) Eigenfunktionen und Eigenwerte
	\begin{equation}
		H\psi_{n,\alpha}\of{\vX} = E_n\psi_{n,\alpha}\of{\vX}
	\end{equation}
	$n$ legt die Eigenwerte $E_n$ fest ($n$, $\alpha$) legen die Eigenzustände $\psi_{n,\alpha}$ fest. $\alpha$ ist ein Entartungsindex, der verschidene Eigenzustände zum selben Eigenwert unterscheidet. Die Indizes ($n$, $\alpha$) können diskret und / oder kontinuierlich verteilt sein. Die Menge der Energieeigenwerte $\curBra{E_n}$ ist das \Hl{Spektrum} des Hamiltonoperators. Das Spektrum kann entsprechend diskret und / oder kontinuierlich sein. [Wir werden alle diese Fälle in Beispielen kennenlernen.]
	\item Die zeitunabhängige Lösung für jede Eigenfunktion ist
	\begin{equation}
		\Psi\of{\vX,t} = \e{-\mI\frac{E_n t}{\hbar}}\psi_{n,\alpha}\of{\vX}
	\end{equation}
	Die \Hl{allgemeine Lösung der Schrödingergleichung} ist eine Superposition der einzelnen Lösungen
	\begin{equation}
		\Psi\of{\vX,t} = \sum\limits_{n,\alpha} c_{n,\alpha} \Psi_{n,\alpha}\of{\vX,t} = \sum\limits_{n,\alpha} c_{n,\alpha} \e{-\mI\frac{E_n t}{\hbar}} \psi_{n,\alpha}
	\end{equation}
	Die $c_{n,\alpha}$ sind die Amplituden für die Eigenzustände $\psi_{n,\alpha}$ und ergeben sich aus der Anfangsbedingung $\Psi\of{\vX,0}$ [siehe später].
	\item Beispiel: Freies Teilchen: Es gilt also $V\of{\vX} \equiv 0$ also ist der Hamiltonoperator $H\of{\vX} = - \frac{\hbar^2}{2 m} \Laplace$ und die zuitunabhängige Schrödingergleichung $H\psi = E \psi$ lautet
	\begin{equation}
		-\frac{\hbar^2}{2m}\Laplace\psi\of{\vX} = E \psi\of{\vX}
	\end{equation}
	Eigenzustände des Hamiltonoperators sind ebene Wellen
	\begin{equation}
		\psi_{\vK}\of{\vX} = \e{\mI\vK\cdot\vX}
	\end{equation}
	mit zugehörigem Eigenwert
	\begin{equation}
		E_{\vK} = \frac{\hbar^2\vK^2}{2m}
	\end{equation}
	Der Wellenvektor $\vK \in \mathds{R}$ legt die Eigenzustände und -werte fest.\\
	Die allgemeine Lösung lautet
	\begin{equation}
		\Psi\of{\vX,t} = \int\dn{k}{3} c\of{\vK} \e{\mI\norBra{\vK\cdot\vX - \omega t}} \quad \text{mit } \omega = \frac{E_{\vK}}{\hbar}
	\end{equation}
	Ist also ein Wellenpaket. Die Amplituden $c\of{\vK}$ ergeben sich aus der Anfangsbedingung $\Psi\of{\vX,0}$.
\end{itemize}

\subsection{Wellenmechanik in einer Dimension mit Potential}

\begin{itemize}
	\item Die Schrödingergleichung soll für stückweise konstantes, zeitunabhägiges Potential $V\of{\vX}$ gelöst werden, also zum Beispiel.
	\begin{figure}[H]
		\centering
		\input{Graph/1_8.tex}
	\end{figure}
	\begin{figure}[H]
		\centering
		\tikzset{thick,every node/.style={scale=1}}
\begin{tikzpicture}
	\fill[white!90!black] (0,0) to (0,5) to (3,5) to (3,0) to (7,0) to (7,5) to (10,5) to (10,0);
  \draw[->] (-0.15,0) to (10.2,0);
  \draw[->] (0,-0.15) to (0,5.2);
  \node[anchor=west] at (10.3,0) {$x$};
  \node[anchor=south] at (0,5.3) {$V\of{x}$};
	\draw[thick] (0,5) to (3,5) to (3,0) to (7,0) to (7,5) to (10,5);
\end{tikzpicture}

	\end{figure}
	Die zeitunabhängige Schrödingergleichung ist
	\begin{equation}
		\norBra{-\frac{\hbar^2}{2m}\nppv{}{x}{2}+V\of{x}} \psi\of{x} = E\psi\of{x}
	\end{equation}
  Wie verhält sich $\psi\of{\vX}$ an den Sprungstellen von $V\of{x}$? Sei $x_0$ eine der Sprungstellen
	\begin{figure}[H]
		\centering
		\tikzset{thick,every node/.style={scale=1}}
\begin{tikzpicture}
	\fill[white!90!black] (0,1) to (5,1) to (5,5) to (10,5) to (10,0) to (0,0);
  \draw[->] (-0.15,0) to (10.2,0);
  \draw[->] (0,-0.15) to (0,5.2);
  \node[anchor=west] at (10.3,0) {$x$};
  \node[anchor=south] at (0,5.3) {$V\of{x}$};
	\draw (-0.15,1) to ++(0.3,0);
	\node[anchor=east] at (-0.25,1) {$V_0$};
	\draw (-0.15,5) to ++(0.3,0);
	\node[anchor=east] at (-0.25,5) {$V_1$};
	\draw[dashed] (0,5) to (5,5);
	\draw (5,-0.15) to ++(0,0.3);
	\node[anchor=north] at (5,-0.25) {$x_0$};
	\draw[dashed] (5,0) to (5,1);
	\draw[thick] (0,1) to (5,1) to (5,5) to (10,5);
\end{tikzpicture}

	\end{figure}
  \begin{equation}
    \nppv{}{x}{2} \psi\of{x} = \frac{2m}{\hbar^2}\of{V\of{x} - E}\psi\of{x}
  \end{equation}
  Integrieren im Intervall $\edgBra{x_0 - \epsilon,x_0+\epsilon}$ ergibt
  \begin{align}
    \int\limits_{x_0-\epsilon}^{x_0+\epsilon} \dd{x} \nppv{}{x}{2} \psi\of{x} & = \frac{2m}{\hbar^2} \int\limits_{x_0-\epsilon}^{x_0+\epsilon} \dd{x} \of{V\of{x} - E} \psi\of{x} \\
    \ppv{}{x} \psi\of{x_0+\epsilon} - \ppv{}{x}\psi\of{x_0-\epsilon} & = \frac{2m}{\hbar^2} \edgBra{\int\limits_{x_0-\epsilon}^{x_0} \dd{x} \of{V_0-E}\psi\of{x} + \int\limits_{x_0}^{x_0+\epsilon}\dd{x} \of{V_1-E}\psi\of{x}} \\
    & = \frac{2m}{\hbar^2} \edgBra{\of{V_0-E} \int\limits_{x_0-\epsilon}^{x_0} \dd{x} \psi\of{x} + \of{V_1 - E}\int\limits_{x_0}^{x_0+\epsilon} \dd{x} \psi\of{x}}
  \end{align}
  Im Limes $\epsilon\to 0$ ist:
  \begin{equation}
    \ppv{}{x} \psi\of{x_0+\epsilon} - \ppv{}{x}\psi\of{x_0-\epsilon}\to 0
  \end{equation}
  Die Ableitung des Wellenfunktion an der Sprungstelle des Potentials ist \Hl{stetig}. Damit ist auch die Wellenfunktion selbst an der Sprungstelle stetig. An Sprungstellen $x_0$ von $V\of{x}$ gilt:
  \begin{empheq}[box=\widefbox]{align}   
    \lim\limits_{\epsilon\to 0} \psi\of{x_0+\epsilon} & = \lim\limits_{\epsilon\to 0} \psi\of{x_0-\epsilon} \\
    \lim\limits_{\epsilon\to 0} \ppv{}{x} \psi\of{x_0+\epsilon} & = \lim\limits_{\epsilon\to 0} \ppv{}{x} \of{x_0-\epsilon}
  \end{empheq}
\end{itemize}

\subsubsection{Potentialstufe}

\begin{itemize}
  \item Gegeben sei ein Potential der Form
  \begin{figure}[H]
    \centering
    \tikzset{thick,every node/.style={scale=1}}
\begin{tikzpicture}
	\fill[white!90!black] (0,0) to (5,0) to (5,5) to (0,5) to (0,0);
  \draw[->] (-5.1,0) to (5.1,0);
  \draw[->] (0,-0.15) to (0,5.2);
  \node[anchor=west] at (5.2,0) {$x$};
 	\node[anchor=south] at (0,5.3) {$V\of{x}$};
	\draw (-0.15,5) to ++(0.3,0);
	\node[anchor=east] at (-0.25,5) {$V_0$};
	\node[anchor=north] at (0,-0.25) {$x_0$};
	\draw[thick] (-5,0) to (0,0) to (0,5) to (5,5);
\end{tikzpicture}

  \end{figure}
  \begin{equation}
    V\of{x} = \begin{faelle} 0 & x \leq 0 \\ V_0 & x > 0 \end{faelle}
  \end{equation}
  Wir suchen die Eigenfunktionen und Eigenwerte der zeitunabhängigen Schrödingergleichung
  \begin{equation}
    \of{-\frac{\hbar^2}{2m}\nppv{}{x}{2} + V\of{x}}\psi\of{x} = E\psi\of{x}
  \end{equation}
  das heißt in den Bereichen \eqref{eq:gebiet1} und \eqref{eq:gebiet2} gilt:
  \begin{align}
    \nppv{}{x}{2} \psi\of{x} & = - \frac{2m}{\hbar^2} E \psi\of{x} \label{eq:gebiet1} \ref{I} \\
    \nppv{}{x}{2} \psi\of{x} & = - \frac{2m}{\hbar^2} \of{E-V_0} \psi\of{x} \label{eq:gebiet2} \ref{II}
  \end{align}
  \item \Hl{1. Fall:} Das Teilchen habe eine \Hl{Energie oberhalb der Potentialstufe}, $E > V_0$.
  \begin{equation}
    k = \sqrt{\frac{2mE}{\hbar^2}} \quad q \mDef \sqrt{\frac{2m\of{E-V-0}}{\hbar^2}} < k
  \end{equation}
  Die Lösungen in \eqref{eq:gebietLsg1} und \eqref{eq:gebietLsg2} sind dann
  \begin{align}
    \psi\of{x} & = A\e{\mI k x} + B\e{- \mI k x} \label{eq:gebietLsg1} \tag{I} \\
    \psi\of{x} = C \e{\mI q x} + D \e{-\mI q x} \label{eq:gebietLsg2} \tag{II}
  \end{align}
  mit beliebigen Amplituden $A,B,C,D$
  \item Die Anschlussbedingungen an den Sprungstellen sind
  \begin{align}
    A + B & = C + D \\
    \mI k \of{A - B} & = \mI q \of{C - D}
  \end{align}
  Die Amplituden $B,C$ der auslaufenden Wellen sollen in Abhängigkeit von den Amplituden der einlaufenden Wellen $A,D$ bestimmt werden:
  \begin{align}
    B - C & = - A + D \\
    k B + q C & = k A + q D \\
    \matTwo{1}{-1}{k}{q} \twoArr{B}{C} & = \matTwo{-1}{1}{k}{q} \twoArr{A}{D} \\
    \twoArr{B}{C} & = \matTwo{1}{-1}{k}{q}^{-1}\matTwo{-1}{1}{k}{q}\twoArr{A}{D} \\
    & = \frac{1}{k+q} \matTwo{q}{1}{-k}{1}\matTwo{-1}{1}{k}{q}\twoArr{A}{D}
  \end{align}
  \item Wir betrachten den Fall eines von links einlaufenden Teilchen, das heißt $A\neq 0$ und $D = 0$ (sodass $j_A \neq 0$und $j_D = 0$). Damit ist
  \begin{equation}
    B = \frac{k-q}{k+q} A \quad C = \frac{2k}{k+q} A
  \end{equation}
  Damit sind die Ströme
  \begin{align}
    j_A & = \frac{\hbar k}{m} \Norm{A}^2 \\
    j_B & = -\frac{\hbar k}{m}\of{\frac{k-q}{k+q}}^2\Norm{A}^2 = - \of{\frac{k-q}{k+q}}^2 j_A \\
    j_C & = \frac{\hbar q}{m}\of{\frac{2k}{k+q}}^2 \Norm{A}^2 = \frac{4kq}{\of{k+q}^2} j_A
  \end{align}
  \item Die \Hl{Reflexions- und Transmissionskoeffizienten} für ein von links mit einem Impuls $p = \hbar k$ (beziehungsweise mit einer kinetischen Energie $E = \frac{\hbar^2 k^2}{2m} > V_0$) einfallenden Teilchens sind
  \begin{align}
    R & = \Norm{\frac{j_B}{j_A}} = \of{\frac{k-q}{k+q}}^2 \quad q = \sqrt{k^2 - \frac{2mV_0}{\hbar^2}} \\
    T & = \Norm{\frac{j_C}{j_A}} = \frac{4kq}{\of{k+q}^2}
  \end{align}
  (Annahme: $k > \frac{2mV_0}{\hbar}$)\\
  Es gilt $R+ T = 1$
  \begin{figure}[H]
    \centering
    \tikzset{thick,every node/.style={scale=1}}
\begin{tikzpicture}[declare function = {
	refl(\t) = ((\t-sqrt(\t^2 - 1))/(\t+sqrt(\t^2 - 1)))^2;
	tran(\t) = (4*\t*sqrt(\t^2 - 1))/((\t + sqrt(\t^2 - 1))^2);
}]
  \draw[->] (-0.15,0) to (10.2,0);
  \draw[->] (0,-0.15) to (0,5.2);
  \node[anchor=west] at (10.3,0) {$k\edgBra{\sqrt{\frac{2mV_0}{\hbar^2}}}$};
 	\node[anchor=south] at (0,5.3) {$T$, $R$};
 	\draw (0.5,-0.15) to ++(0,0.3);
 	\node[anchor=north] at (0.5,-0.25) {$1.0$};
 	\node[anchor=east] at (-0.25,0) {$0.0$};
 	\draw (-0.15,5) to ++(0.3,0);
 	\node[anchor=east] at (-0.25,5) {$1.0$};
 	\draw (9.5,-0.15) to ++(0,0.3);
 	\node[anchor=north] at (9.5,-0.25) {$1.2$};
	\draw[thick,smooth,samples=200,domain=0.0001:9] plot(\x+0.5,{5*tran((\x/45+1))});
	\node at (5,3.5) {$T$};
	\draw[thick,smooth,samples=200,domain=0.0001:9] plot(\x+0.5,{5*refl((\x/45+1))});
	\node at (5,1.5) {$R$};
\end{tikzpicture}

  \end{figure}
  Für Energien $E > V_0$ gibt es eine endliche Wahrscheinlichkeit, reflektiert zu werden!
  \item Das \Hl{Eigenwertproblem}
  \begin{equation}
    H\psi_k\of{x} = E_k \psi_k\of{x} \quad E_k = \frac{\hbar^2 k^2}{2 m}
  \end{equation}
  ist (für Wellen mit $E > V_0$ und $D = 0$, $A = 1$) gelöst mit 
  \begin{equation}
    \psi_k = \begin{faelle} \e{\mI k x} + \frac{k-q}{k+q} \e{-\mI k x} & x < 0 \\ \frac{2 k}{k + q} \e{\mI q x} & x \geq 0 \end{faelle}
  \end{equation}
  Durch Superposition dieser Wellen können von links einlaufende Wellenpakete konstruiert werden
  \begin{equation}
    \Psi\of{x,t} = \frac{1}{\sqrt{2\pi}} \int\dd{k} g\of{k} \e{-\mI w_k t} \psi_k\of{x}
  \end{equation}
  \item \Hl{2. Fall:} Das Teilchen habe eine \Hl{Energie unterhalb der Potentialstufe}, $E < V_0$. \\
  Die Schrödingergleichung ist nun
  \begin{align}
    \nppv{}{x}{2} \psi\of{x} & = - \frac{2m}{\hbar^2}  \psi\of{x} \\
    \nppv{}{x}{2} \psi\of{x} & = \frac{2m}{\hbar^2} \of{V_0 - E} \psi\of{x} 
  \end{align}
  Mit 
  \begin{equation}
    k = \sqrt{\frac{2mE}{\hbar^2}} \quad \mu \mDef \sqrt{\frac{2m\of{V_0-E}}{\hbar^2}}
  \end{equation}
  lauten die Lösungen in (I) und (II)
  \begin{align}
    \psi\of{x} & = A \e{\mI k x} + B \e{-\mI k x} \\
    \psi\of{x} & = C \e{-\mu x} + D \e{\mu x}
  \end{align}
  Die Lösungen in (I) sind wieder ebene Wellen. In (II) ergeben sich Exponentialfunktionen.
  \item Die Komponente $\e{+\mu x}$ divergiert für $x\to \infty$. Die Wellenfunktion muss normierbar sein. Also fordern wir $D\equiv 0$. Die Anschlussbedingungen bei $x = 0$ liefern dann für den Fall eines von links einlaufenden Teilchens
  \begin{equation}
    B = \frac{k - \mI \mu}{k + \mI \mu} A \quad C = \frac{2k}{k+\mI \mu} A
  \end{equation}
  Die Energieeigenfunktion zum Energieeigenwert $E_k = \frac{\hbar^2\vK^2}{2m}$ lauten damit ($A = 1$)
  \begin{equation}
    \psi_k\of{x} = \begin{faelle} \e{\mI k x} + \frac{k - \mI \mu}{k + \mI \mu} \e{-\mI k x} & x \leq 0 \\ \frac{2k}{k+\mI \mu} \e{-\mu x} & x > 0 \end{faelle}
  \end{equation}
  \item Die Wahrscheinlichkeitsdichte in $x > 0$ ist 
  \begin{equation}
    \Norm{\psi_k\of{x}}^2 = \frac{4k^2}{k^2 + \mu^2} \e{-2\mu x} = \frac{4 E}{V_0} \e{-2\mu x}
  \end{equation}
  Die \Hl{Eindringtiefe} ist
  \begin{equation}
    \frac{1}{\mu} = \frac{\hbar}{\sqrt{2m\of{V_0-E}}}
  \end{equation}
  Die Eindringtiefe ist klein für große Massen und hohe Potentiale $V_0 \gg E$.
  \item Insbesondere für eine \Hl{unendlich hohe Potentialstufe} $V_0 \to \infty$ gilt
  \begin{equation}
    \psi_k\of{x} \propto \begin{faelle} \sin k x & x < 0 \\ 0 & x \geq 0 \end{faelle}
  \end{equation}
  Das heißt, die Randbedingung für eine unendlich hohe Potentialstufe bei $x_0$ ist 
  \begin{equation}
    \psi\of{x_0} = 0
  \end{equation}
  Die Ableitung der Wellenfunktion ist in $x_0$ unstetig!
\end{itemize}

\subsubsection{Potentialschwelle}
\begin{itemize}
  \item Wir betrachten ein Potential der Form 
  \begin{equation}
    V\of{x} = \begin{faelle} 0 & x \leq 0 \\ V_0 & 0 \leq x \leq l \\ 0 & l \leq x \end{faelle}
  \end{equation}
  \begin{figure}[H]
    \centering
    \tikzset{thick,every node/.style={scale=1}}
\begin{tikzpicture}
	\fill[white!90!black] (0,1) to (5,1) to (5,5) to (10,5) to (10,0) to (0,0);
  \draw[->] (-0.15,0) to (10.2,0);
  \draw[->] (0,-0.15) to (0,5.2);
  \node[anchor=west] at (10.3,0) {$x$};
  \node[anchor=south] at (0,5.3) {$V\of{x}$};
	\draw (-0.15,1) to ++(0.3,0);
	\node[anchor=east] at (-0.25,1) {$V_0$};
	\draw (-0.15,5) to ++(0.3,0);
	\node[anchor=east] at (-0.25,5) {$V_1$};
	\draw[dashed] (0,5) to (5,5);
	\draw (5,-0.15) to ++(0,0.3);
	\node[anchor=north] at (5,-0.25) {$x_0$};
	\draw[dashed] (5,0) to (5,1);
	\draw[thick] (0,1) to (5,1) to (5,5) to (10,5);
\end{tikzpicture}

  \end{figure}
  \item \Hl{1. Fall:} Das Teilchen habe eine Energie \Hl{oberhalb der Potentialschwelle}, $E > V_0$. Die Lösung der Schrödingergleichung in den $3$ Bereichen ist
  \begin{align}
    \psi\of{x} & = A \e{\mI k x} + B \e{-\mI k x} \\
    \psi\of{x} & = C \e{\mI q x} + D \e{-\mI q x} \\
    \psi\of{x} & = F \e{\mI k x} + G \e{-\mI k x}
  \end{align}
  wobei
  \begin{equation}
    k = \sqrt{\frac{2mE}{\hbar^2}} \quad q \mDef \sqrt{\frac{2m\of{E-V_0}}{\hbar^2}} < k
  \end{equation}
  \begin{figure}[H]
    \centering
    \tikzset{thick,every node/.style={scale=1}}
\begin{tikzpicture}
	\fill[white!90!black] (0,0) to (5,0) to (5,5) to (0,5) to (0,0);
  \draw[->] (-5.1,0) to (5.1,0);
  \draw[->] (0,-0.15) to (0,5.2);
  \node[anchor=west] at (5.2,0) {$x$};
 	\node[anchor=south] at (0,5.3) {$V\of{x}$};
	\draw (-0.15,5) to ++(0.3,0);
	\node[anchor=east] at (-0.25,5) {$V_0$};
	\node[anchor=north] at (0,-0.25) {$x_0$};
	\draw[thick] (-5,0) to (0,0) to (0,5) to (5,5);
\end{tikzpicture}

  \end{figure}
  \item Die Anschlussbedingungen erfordern Stetigkeit von $\psi\of{x}$ und $\psi'\of{x}$ bei $x = 0$ und $x = l$
  \begin{align}
    A + B & = C + D \\
    k\of{A - B} & = q \of{C - D} \\
    C \e{\mI q l} + D \e{-\mI q l} & = F \e{\mI k l} + G \e{-\mI k l} \\
    q\of{C \e{\mI q l} - C \e{-\mI q l}} & = k\of{F \e{\mI k l} - G \e{-\mI k l}}
  \end{align}
  Die auslaufenden Wellen $B, F$ sollen für gegebene einlaufende Wellen $A,G$ berechnet werden. Wir können uns auf ein von links einlaufendes Teilchen spezialisieren, $G = 0$. \\
  Die Lösung (mit Matrixmethode, wie vorher) ist:
  \begin{align}
    F & = \edgBra{\cos ql - \mI \frac{k^2 + q^2}{2kq} \sin ql}^{-1} \e{-\mI k l} A \\
    B & = \mI \frac{q^2 - k^2}{2kq} \sin ql \e{\mI k l} F
  \end{align}
  \item Der transmittierte/reflektierte Strom ist
  \begin{equation}
    j_F = \frac{\hbar k}{m} \Norm{F}^2 \quad j_B = \frac{\hbar k}{m} \Norm{B}^2
  \end{equation}
  und der \Hl{Reflexions- und Transmissionskoeffizient}
  \begin{empheq}[box=\widefbox]{align}
    R = \frac{j_B}{j_A} & = \frac{\of{k^2 - q^2}^2 \sin^2 ql}{4k^2 q^2 + \of{k^2 - q^2}^2 \sin^2 ql} \\
    T = \frac{j_F}{j_A} & = \frac{4k^2 q^2}{4k^2 q^2 + \of{k^2 - q^2}^2 \sin^2 ql}
  \end{empheq}
  \Hl{Diskussion:}
  \begin{itemize}
    \item Es gilt $R + T = 1$.
    \item Transmissionskoeffizient gegen Breite $l$ der Schwelle (für feste Energie $E = \frac{\hbar^2k^2}{2m}$ des Teilchens und Potentialhöhe $V_0$)
    \item Die Transmission besitzt sogenannte \Hl{Streuresonanzen} bei 
    \begin{equation}
      \sin q l \quad \text{also} \quad ql = n \pi \quad n\in \mathbb{N}
    \end{equation}
    also für eine Wellenzahl $q = \frac{n\pi}{l}$. Im Bereich der Barrieren entstehen dann stehende Wellen.
  \end{itemize}
  \item \Hl{2.Fall:} Wir betrachten ein Teilchen mit \Hl{Energie unterhalb der Potentialschwelle}, $E < V_0$. Die Lösung in (II) ist nun (wie vorher)
  \begin{equation}
    \psi\of{x} = C \e{-\mu x} + D\e{+\mu x}
  \end{equation}
  Die Lösung ist die gleiche wie für $E > V_0$ mit $q \to \mI \mu$, $\mu = \sqrt{\of{V_0 - E} \frac{2m}{\hbar^2}}$. Der Reflexions- und Transmissionskoeffizient ist
  \begin{align}
    R & = \frac{\of{k^2 + \mu^2}^2 \sinh^2 \mu l}{4k^2 \mu^2 + \of{k^2 + \mu^2}^2 \sinh^2 \mu l} \\
    T & = \frac{4k^2 \mu^2}{4k^2 \mu^2 + \of{k^2 + \mu^2}^2 \sinh^2 \mu l}
  \end{align}
  \Hl{Diskussion:}
  \begin{itemize}
    \item $T \neq 0$ sogar für $E < V_0$: \Hl{Tunneleffekt}
    \item Für kleine Energien $E$ und lange Barrieren $l$, so dass
    \begin{equation}
      \mu l = \sqrt{\frac{2m\of{V_0-E}}{\hbar^2}} l \gg 1
    \end{equation}
    gilt
    \begin{equation}
      T \approx \frac{16\of{E-V_0}}{V_0}^2 \e{- 2 \mu l}
    \end{equation}
    Die Eindringtiefe ist wieder
    \begin{equation}
      \frac{1}{\mu} = \frac{\hbar}{\sqrt{2m\of{V_0 - E}}}
    \end{equation}
    \item \Hl{Beispiel 1:} Für ein Elektron mit Energie $E = 1 \unit{eV}$, $V_0 = 2 \unit{eV}$ ist die Eindringtiefe $\mu^{-1} \approx 2 \cdot 10^{-10} \unit{m} = 2 \unit{\AA}$. Bei einer Barriere mit $l = 1 \unit{\AA}$ ist die Transmissionswahrscheinlichkeit $T = 0.78$.
    \item \Hl{Beispiel 2:} Für ein Proton ($m_p = 1840 m_e$) mit $E = 1 \unit{eV}$ und $V_0 = 2 \unit{eV}$ ist $\mu^{-1} = 5 \cdot 10^{-12} \unit{m}$ und für $l = 1 \unit{\AA}$ ist $T = 4\cdot 10^{-19}$
  \end{itemize}
\end{itemize}

\subsubsection{Der Potentialtopf}
\begin{itemize}
  \item Wir betrachten ein Potential der Form
  \begin{equation}
    V\of{x} = \begin{faelle} -V_0 & \Norm{x} \leq l \\ 0 & \Norm{x} > l \end{faelle}
  \end{equation}
  Die zeitunabhängige Schrödingergleichung
  \begin{equation}
    \of{-\frac{\hbar^2}{2m} \nppv{}{x}{2} + V\of{x}}\psi\of{x} = E \psi\of{x}
  \end{equation}
  Die Zustände mit $E > 0$ entsprechenim wesentlichen denen der Potentialschwelle für ($E > V_0$). Welche Zustände mit $E < 0$ gibt es im Potentialtopf?
  \item Für $-V_0 < E < 0$ sind die Lösungen in den $3$ Bereichen
  \begin{align}
    \psi\of{x} & = A \e{\mu x} \\
    \psi\of{x} & = B \cos kx + C\sin kx \\
    \psi\of{x} & = D \e{-\mu x} 
  \end{align}
  mit:
\end{itemize}