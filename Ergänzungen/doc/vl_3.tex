Zerlegung von halbeinfachen Lie-Algebren in die direkte Summe von einfachen Lie-Algebren. Sei $L$ halbeinfache Lie-Algebra und $I \subset L$ Ideal 
\begin{equation}
	K\of{\eof{I^{\perp},I},L} = K\of{I^{\perp},\eof{I,L}} = K\of{I^{\perp},I} = 0
\end{equation}
Dann $\eof{I^{\perp},I} = N\of{L} = \cof{0}$ und damit $I^{\perp} \cap I = \cof{0}$, also $L = I \oplus I^{\perp}$. Enthält $I$ weitere Ideale kann die Zerlegung analog weiter geführt werden, bis keine weiteren Ideale mehr existieren:
\begin{equation}
	L = \bigoplus\limits_{i=1}^n I_i
\end{equation}
\begin{proposition}
	$L$ halbeinfach, dann 
	\begin{equation}
		\eof{L,L} = \spann\sDef{\eof{x,y}}{x,y \in L}
	\end{equation}
\end{proposition}
\begin{definition}
	Eine Lie-Algebra für die $\eof{L,L} = L$ gilt, heißt perfekt
\end{definition}
\begin{definition}
	Eine Lie-Algebra heißt \hl{kompakt}, falls $K$ negativ definit ist. Achtung: Das Wort \qt{Kompakt} bezieht sich auf die zur Lie-Algebra zugehörigwn Lie-Gruppen. EIne Lie-Algebra als topologischer Raum ist natürlich nie kompakt.
\end{definition}
\begin{beispiel}
	$L = \of{\bR^3,\times}$, $\idx{\vE}{_a}\times\idx{\vE}{_b} = \idx{\epsilon}{_a_b^c} \idx{\vE}{_c}$, $\idx{C}{^c_a_b} = \idx{\epsilon}{_a_b^c}$
	\begin{equation}
		\idx{K}{_a_b} = \idx{C}{^n_a_m} \idx{C}{^m_b_n} = \idx{\epsilon}{_a_m^n} \idx{\epsilon}{_b_n^m} = -2 \idx{\delta}{_a_b}
	\end{equation}
\end{beispiel}

\subsection{Matrix Lie-Gruppen}
\qt{Matrix} heißt: Jede der betrachteten Gruppen besitzt eine treue endliche Darstellung \kla{\sog \qt{definierende Darstellung}}. Achtung: Es existieren endlich \dims Lie-Gruppen die keine Matrixgruppen sind, \exmpl alle Überlagerungsgruppen von $\opSL\of{2,\bR}$
\begin{beispiel}
	\begin{gather}
		\opGL\of{\bF^n} \mDef \sDef{x \in \eMo\of{\bF^n}}{\det\of{x} \neq 0} \\
		\opSL\of{\bF^n} \mDef \sDef{x \in \opGL\of{\bF^n}}{\det\of{x} = 1} \\
		\opO\of{p,q_{-}} \mDef \sDef{x \in \eMo\of{\bF^n}}{x E^{\of{p,q}} x^T = E^{\of{p,q}}} \\
		\opSO\of{p,q_{-}} \mDef \sDef{x \in \opO\of{p,q_{-}}}{\det\of{x} = 1} \\
		\opU\of{p,q_{-}} \mDef \sDef{x \in \opGL\of{\bC^n}}{x E^{\of{p,q}} X^{T} = E^{\of{p,q}}} \\
		\opSU\of{p,q_{-}} \mDef \sDef{x \in \opU\of{p,q_{-}}}{\det\of{x} = 1} \\
		\opSO\of{1,3} = \text{Lorentzgruppe} \cup \cof{-\one_4}
	\end{gather}
	wobei
	\begin{equation}
		E^{\of{p,q}} = \of{\begin{array}{c|c} \one_p & 0 \\\midrule 0 & -\one_q\end{array}}
	\end{equation}
\end{beispiel}
Ebenfalls Matrix-Gruppen sind solche, die aus semi-direkten Produkten mit $\bF^n$ entstehbar. \\
Sei $G \subset\eMo\of{V}$ \kla{$V \cong \bF^n$} eine Gruppe \& $\fDef{A}{\bR\supset \of{-\epsilon,\epsilon}}{G}$ differenzierbare Kurve mit $A\of{0} = \id$. Wir definieren $\dot{A} \mDef \evLi{\dd{s}}_{s=0} A\of{s} =$ Tangentialvektor an der Gruppenidentität.
\begin{satz}
	Die Menge der Tangentialvektoren an die Gruppenidentität bilden eine \hl{reelle} Lie-Algebra, $\lie\of{G}$.
\end{satz}
\begin{proof}
	\begin{enumerate}
		\item Linearität: Ist $X = \dot{A}$ und $Y = \dot{B}$, definiere $C\of{s} = A\of{s}B\of{s}$, dann $\dot{C} = \dot{A} + \dot{B} = X + Y$. Ebenso: Ist $X = \dot{A}$, definiere $B\of{s} = A\of{as}$ $\dot{B} = a X \fuerall a \in \bR$. Geschwindigkeit bei $e\in G$ bildet Vektorraum über $\bR$.
		\item Abgeschlossenheit unter Kommutatorbildung: Sei $X = \dot{A}$ \& $Y = \dot{B}$. Wir müssen zeigen, dass eine Kurve $C\of{s}$ existiert mit $C\of{0} = e$ und $\dot{C} = \eof{X,Y} = XY - YX$. Definiere also 
		\newcommand{\oftau}{\of{\tau\of{s}}}
		\begin{equation}
			C\of{s} = \begin{faelle} A\oftau B\oftau A^{-1}\oftau B^{-1}\oftau & s \geq 0 \\ B\oftau A\oftau B^{-1}\oftau A^{-1}\oftau & s\leq 0 \end{faelle}
		\end{equation}
		wobei $\tau\of{s} = \sign\of{s} \sqrt{s}$ und invers $s\of{\tau} = \sign\of{\tau}\tau^2$
		\begin{figure}[H]
			\centering
			\begin{tikzpicture}
				\draw[->] (-5,0) to (5,0);
				\draw[->] (0,-2.5) to (0,2.5);
				\node[draw=none] at (5.2,0) {$s$};
				\node[draw=none] at (0,2.7) {$\tau$};
				\draw[domain=0:1,smooth] (-5,-2.5) plot ({-\x*\x*5},{-\x*2.5}) plot ({\x*\x*5},{\x*2.5});
			\end{tikzpicture}
		\end{figure}
		Obwohl keine der Kurven $s \mapsto A\oftau$ \etc selber differenzierbar ist \kla{weil $\tau\of{s}$ nicht differenzierbar ist}, ist dennoch die Kurve $s\mapsto C\of{s}$ bei $s = 0$ differenzierbar. Für $s \searrow 0$ \kla{Rechtsableitung} gilt:
		\begin{equation}
			\begin{split}
				\dot{C}_R = \lim\limits_{s\searrow 0}\cof{\frac{C\of{s}-e}{s}} & = \lim\limits_{s\searrow 0} \cof{\frac{\eof{A\oftau,B\oftau}A^{-1}\oftau B^{-1}\oftau}{s}} \\
				& = \lim\limits_{\tau\searrow 0} \cof{\eof{\frac{A\of{\tau}-e}{\tau},\frac{B\of{\tau}-e}{\tau}}A^{-1}\of{\tau}B^{-1}\of{\tau}} \\
				& = \eof{X,Y}
			\end{split}
		\end{equation}
		$\dot{C}_L$ analog. 
	\end{enumerate}
\end{proof}
	Da die Lie-Struktur durch die von $\eMo\of{V}$ induziert wird, gilt automatisch die Jacobi-Identität. \\
	Ist $D \in \Hom\of{G,\opGL\of{W}}$ eine lineare Darstellung von $G$ auf den Vektorraum $W$, dann induziert diese eindeutig eine Darstellung
	\begin{equation}
		D_{*} \in \Hom\of{\lie\of{G},\eMo\of{W}}
	\end{equation}
\begin{proof}
	Das sieht man so: Sei $A\of{s}$ Kurve in $G$ mit $A\of{0} = e$ und $\evLi{\dd{s}}_{s=0} A\of{s} = X$. Dann ist $A'\of{s} = \of{D\circ A}\of{s} = D\of{A\of{s}}$ eine Kurve in $\opGL\of{W}$ mit $A'\of{0} = \evLi{e}_{\opGL\of{W}}$. Wir sehen voraus, dass $D$ differenzierbar ist. 
	\begin{equation}
		\evLi{\dd{s}}_{s=0} A'\of{s} = D_{*}\of{X}.
	\end{equation}
	Dabei ist $D_{*}$ die Ableitung der \abb $D$ bei $e \in G$. Wir müssen zeigen:
	\begin{equation}
		D_{*}\of{\eof{X,Y}} = \eof{D_{*}\of{X},D_{*}\of{Y}}
	\end{equation}
	Wir zuvor betrachten wir Kurven in $G$; $A\of{s}$ $B\of{s}$ mit $\dot{A} = X$ $\dot{B} = Y$ und $C\of{s}$ \defs wie oben und deren Bilder unter $D$ in $\opGL\of{W}$ \kla{$A',B',C'$}. Für $s \geq 0$ ist dann:
	\begin{equation}
		C'\of{s} = D\of{C\of{s}}
	\end{equation}
	Abgeleitet ergibt einerseits
	\begin{equation}
		\dot{C}' = D_{*} \dot{C} = D_{*}\of{\eof{X,Y}}
	\end{equation}
	\newcommand{\oftau}{\of{\tau\of{s}}}
	andererseits, weil $D$ Homomorphismus 
	\begin{equation}
		C' = D\of{A\oftau}D\of{B\oftau}\eof{D\of{A\oftau}}^{-1}\eof{D\of{A\oftau}}^{-1}
	\end{equation}
	folgt nach identischer Rechnung $\dot{C}' = \eof{D_{*}X,D_{*}Y}$
\end{proof}
Adjungierte Darstellung
\begin{gather}
 \Ad \in \Hom\of{G,\opGL\of{\lie\of{G}}} \\
 G \owns A \mapsto \Ad_A : x \mapsto \Ad_A \of{x}
\end{gather}
Sei $B\of{s}$ Kurve in $G$ mit $B\of{0} = e$ und $\dot{B} = \evLi{\dd{s}}_{s=0} B\of{s} = Y$, dann 
\begin{equation}
	\Ad_A\of{Y} \mDef \evLi{\dd{s}}_{s=0} A B\of{s} A^{-1} = A Y A^{-1}
\end{equation}
Achtung: Da $Y \in \eMo\of{V}$ \iall nicht in $\opGL\of{V}$ liegt, ist hier mit \qt{$\cdot$} das Produkt in der assoziativen Algebra $\eMo\of{V}$ gemeint \kla{Komposition}. Ist nun $A = A\of{t}$ selbst eine Kurve mit $A\of{0} = e$ und $\dot{A} = X$, dann folgt mit 
\begin{equation}
	\evLi{\dd{t}}_{t=0}A\of{t} = - X
\end{equation}
\begin{equation}
	\begin{split}
		\evLi{\dd{t}}_{t=0} \Ad_{A\of{t}} \of{Y} & = \Ad_{*} \of{X}\of{Y} \\
		& = \eof{X,Y} = \ad_X\of{Y}
	\end{split}
\end{equation}
Damit $\Ad_{*} = \ad$

\subsection{Die Exponentialabbildung}
Sei $\exp : \eMo\of{V} \to \eMo\of{V}$ \defs durch 
\begin{equation}
	\exp\of{x} \mDef \sum\limits_{n=0}^\infty \frac{X^n}{n!} \quad X^n = X \circ \dots \circ X
\end{equation}
Ist $X \in \eMo\of{V}$ und $A \in \opGL\of{V}$ und sei $\Ad_A : \eMo\of{V} \to \eMo\of{V}$ die \abb 
\begin{equation}
	\Ad_A\of{X} = A \circ X \circ A^{-1} \quad \Ad_A\of{X^n} = \eof{\Ad_A\of{X}}^n
\end{equation}
Damit erhält man $\Ad_A \circ \exp = \exp \circ \Ad_A$
\begin{proposition}
	$\det\circ \exp = \exp \circ \spur$
\end{proposition}
\begin{proof}
 	einfach über $\bC$ \kla{nötigenfalls Komplexifizierung $V \otimes \bC$}, dann existiert eine Basis $\sDef{\idx{e}{_a}}{a = 1,\dots,n}$ aus Eigenvektoren von $X \in \eMo\of{V^{\bC}}$
 	\begin{equation}
 		X = \dotMat{\lambda_1}{*}{0}{\lambda_n} \quad \exp\of{X} = \dotMat{\mEx{\lambda_1}}{*}{0}{\mEx{\lambda_1}}
 	\end{equation}
 	\begin{equation}
 		\begin{split}
 			\det\of{\exp\of{X}} & = \exp\of{\sum\limits_n \lambda_n} \\
 			& = \exp\of{\spur\of{X}}
 		\end{split}
 	\end{equation}
\end{proof}
\begin{korollar}
	\begin{equation}
		\exp\of{\eMo\of{V}} \subseteq \opGL^{+}\of{V} = \sDef{x \in \opGL\of{V}}{\det\of{X} > 0}
	\end{equation}
\end{korollar}
\begin{proof}
	Ist $\fDef{X}{\of{-\epsilon,\epsilon}}{\eMo\of{V}}$ mit $X\of{0} = 0$ und $\dot{X} = \evLi{\dd{s}}_{s=0} X\of{s}$, dann $A\of{s} \mDef \exp\of{X\of{s}}$ mit $A\of{0} = e$
	\begin{equation}
		\dot{A} = \evLi{\dd{t}}_{t=0}\exp\of{X\of{t}} = \exp_{*} \of{\dot{X}} = \dot{X} \fuerall X
	\end{equation}
	Nach dem Satz über Umkehrfunktionen bildet $\exp$ eine offene Umgebung $U$ von $O \in \eMo\of{V}$ in eine offene Umgebung $U'$ von $e \in \opGL\of{V}$ ab, und zwar bijektiv und in beide Richtungen \difb. $\exp$ ist also lokal ein Diffeomorphismus. Global kann dagegen sowohl die Injektivität, als auch die Surjektivität nicht erfüllt sein.
\end{proof}
\begin{beispiel} 
	Exponentialabbildungen
	\begin{enumerate}
		\item $G = \opU\of{1} = \sDef{\mEx{\mIm\lambda}}{\lambda \in \bR}$ und $\lie\of{G} = \sDef{\mIm\lambda}{\lambda\in\bR}$. Kurven in $G$ sind \exmpl $A\of{s} = \mEx{\mIm\lambda\of{s}}$ mit $\lambda\of{0} = 0$, $\dot{A}\of{0} = \mIm\dot{\lambda} \in \mIm \bR$ aber $\exp\of{\mIm\lambda} = \exp\of{\mIm\lambda'} \eqval \lambda - \lambda' = 2 \pi n$, also ist $\exp$ surjektiv, aber nicht injektiv.
		\item $G = \opSO\of{3} = \opSO\of{\bR^3} = \sDef{A \in \opGL\of{\bR^3}}{A^T = A^{-1}}$ und $\lie\of{G} = \sDef{x \in \eMo\of{\bR^3}}{X + X^T = 0}$. $\exp\text{-Funktion}$ ist auch hier nicht injektiv. Auch hier gilt Surjektivität \kla{Übung}. \\
		Das sind aber Spezialfälle. \iall nicht surjektiv.
		\item $\opSL\of{2,\bR}$ Behauptung: Kein Element der Form 
		\begin{equation}
			D_n = \twoMat{-n}{0}{0}{-\frac{1}{n}} \quad n \in \bR_+\setminus \cof{1}
		\end{equation}
		liegt im Bild von $\exp$
	\end{enumerate}
\end{beispiel}
\begin{proof}
	Clevere Beobachtung: Ist $A \in G$ in $\img\of{\exp}$, dann \exs $H \in G$ mit $H^2 = A$; $A = \exp{x}$, $H = \exp{\frac{1}{2} x}$. Es reicht also zu zeigen, dass $D_n$ keine Wurzel hat, \dah $D_n \neq H^2 $
	\begin{equation}
		H = \twoMat{a}{b}{c}{d} \quad H^2 = \twoMat{a^2 + bc}{b\of{a+d}}{c\of{a+d}}{d^2 + bc} \tbE \twoMat{-n}{0}{0}{-\frac{1}{n}}
	\end{equation}
	\begin{gather}
		b\of{a+d} = c\of{a+d} = 0 \\
		a^2 + bc = -n \\
		d^2 + bc = -\frac{1}{n}
	\end{gather}
	\begin{enumerate}
		\item Fall: $\of{a+d} \neq 0$, $b = c = 0$, also $a^2 = -n$
		\item Fall: $\of{a+d} = 0$, also $n^2 = 1$
	\end{enumerate}
\end{proof}
Beachte, dass $D_n$ in der Komponente der Gruppeneinheit liegt, \dah es gibt eine stetige Kurve $A\of{s}$, die $e$ mit $D_n$ verbindet
\begin{equation}
	A\of{s} = \begin{faelle} \twoMat{\cos\of{2\pi s}}{\sin\of{2\pi s}}{-\sin\of{2\pi s}}{\cos\of{2\pi s}} & s \in \eof{0,\frac{1}{2}} \\ \twoMat{\varphi\of{s}}{0}{0}{\frac{1}{\varphi\of{s}}} & s \in \eof{\frac{1}{2},1} \end{faelle}
\end{equation}
mit $\varphi\of{s} = 2\of{1-n} s + \of{n-2}$ sodass $\varphi\of{\frac{1}{2}} = -1$ und $\varphi\of{1} = -n$ \\
Obwohl $\fDef{\exp}{\lie\of{G}}{G}$ \iall nicht surjektiv ist \kla{auch dann nicht, wenn $G$ zusammenhängend ist} wird doch ganz $G$ \kla{zusammenhängend} von $\lie\of{G}$ im folgenden Sinne \qt{erzeugt}:
\begin{proposition}
	Ist $G$ zusammenhängende Lie-Gruppe und $A \in G$. Dann \exs endlich viele $X_1, \dots, X_n \in \lie\of{G}$, dass
	\begin{equation}
		A = \exp\of{X_1}\cdot \dots \cdot \exp\of{X_n}
	\end{equation}
\end{proposition}
Zum Beweis benötigen wir folgendes 
\begin{lemma}
	Sei $G$ zusammenhängende topologische Grupep und $V \subset G$ offene Umgebung der Identität $e \in G$. Sei $A \in G$, dann \exs endlich viele $g_1,\dots, g_n \in V$, $A = g_1 \cdot \dots \cdot g_n$
\end{lemma}
\begin{proof}
	Wir betrachten die Menge aller endlichen Produkte von Elementen aus $V$, also 
	\begin{equation}
		G' = \sDef{g\in G}{\exists g_1,\dots,g_n\in V, n < \infty, g = g_1\cdot\dots\cdot g_n}
	\end{equation}
	$G'$ ist offen und eine Untergruppe. Es ist $V\subset G'$, also $gV = \sDef{gg'}{g'\in V}$ offene Umgebung von $g$ denn die Linksoperation $\fDef{L_g}{G}{G} \quad h \mapsto gh$ hat stetige Inverse $L_{g}^{-1} = L_{g^{-1}}$, somit offene Abbildungen. In einer topologischen Gruppe ist eine offene Untergruppe auch abgeschlossen, da sie das Komplement der Vereinigung aller von $G'$ verschiedenen Nebenklassen $gG'$ ist, die alle offen sind. $G' \subset G$ ist also offen und abgeschlossen. Da $G'$ nichtleer und $G$ zusammenhängend ist, folgt $G' = G$
\end{proof}