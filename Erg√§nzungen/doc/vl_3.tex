Zerlegung von halbeinfachen Lie-Algebren in die direkte Summe von einfachen Lie-Algebren. Sei $L$ halbeinfache Lie-Algebra und $I \subset L$ Ideal 
\begin{equation}
	K\of{\eof{I^{\perp},I},L} = K\of{I^{\perp},\eof{I,L}} = K\of{I^{\perp},I} = 0
\end{equation}
Dann $\eof{I^{\perp},I} = N\of{L} = \cof{0}$ und damit $I^{\perp} \cap I = \cof{0}$, also $L = I \oplus I^{\perp}$. Enthält $I$ weitere Ideale kann die Zerlegung analog weiter geführt werden, bis keine weiteren Ideale mehr existieren:
\begin{equation}
	L = \bigoplus\limits_{i=1}^n I_i
\end{equation}
\begin{proposition}
	$L$ halbeinfach, dann 
	\begin{equation}
		\eof{L,L} = \spann\sDef{\eof{x,y}}{x,y \in L}
	\end{equation}
\end{proposition}
\begin{definition}
	Eine Lie-Algebra für die $\eof{L,L} = L$ gilt, heißt perfekt
\end{definition}
\begin{definition}
	Eine Lie-Algebra heißt \hl{kompakt}, falls $K$ negativ definit ist. Achtung: Das Wort \qt{Kompakt} bezieht sich auf die zur Lie-Algebra zugehörigwn Lie-Gruppen. EIne Lie-Algebra als topologischer Raum ist natürlich nie kompakt.
\end{definition}
\begin{beispiel}
	$L = \of{\bR^3,\times}$, $\idx{\vE}{_a}\times\idx{\vE}{_b} = \idx{\epsilon}{_a_b^c} \idx{\vE}{_c}$, $\idx{C}{^c_a_b} = \idx{\epsilon}{_a_b^c}$
	\begin{equation}
		\idx{K}{_a_b} = \idx{C}{^n_a_m} \idx{C}{^m_b_n} = \idx{\epsilon}{_a_m^n} \idx{\epsilon}{_b_n^m} = -2 \idx{\delta}{_a_b}
	\end{equation}
\end{beispiel}

\subsection{Matrix Lie-Gruppen}
\qt{Matrix} heißt: Jede der betrachteten Gruppen besitzt eine treue endliche Darstellung \kla{\sog \qt{definierende Darstellung}}. Achtung: Es existieren endlich \dims Lie-Gruppen die keine Matrixgruppen sind, \exmpl alle Überlagerungsgruppen von $\opSL\of{2,\bR}$
\begin{beispiel}
	\begin{gather}
		\opGL\of{\bF^n} \mDef \sDef{x \in \eMo\of{\bF^n}}{\det\of{x} \neq 0} \\
		\opSL\of{\bF^n} \mDef \sDef{x \in \opGL\of{\bF^n}}{\det\of{x} = 1} \\
		\opO\of{p,q_{-}} \mDef \sDef{x \in \eMo\of{\bF^n}}{x E^{\of{p,q}} x^T = E^{\of{p,q}}} \\
		\opSO\of{p,q_{-}} \mDef \sDef{x \in \opO\of{p,q_{-}}}{\det\of{x} = 1} \\
		\opU\of{p,q_{-}} \mDef \sDef{x \in \opGL\of{\bC^n}}{x E^{\of{p,q}} X^{T} = E^{\of{p,q}}} \\
		\opSU\of{p,q_{-}} \mDef \sDef{x \in \opU\of{p,q_{-}}}{\det\of{x} = 1} \\
		\opSO\of{1,3} = \text{Lorentzgruppe} \cup \cof{-\one_4}
	\end{gather}
	wobei
	\begin{equation}
		E^{\of{p,q}} = \of{\begin{array}{c|c} \one_p & 0 \\\midrule 0 & -\one_q\end{array}}
	\end{equation}
\end{beispiel}
Ebenfalls Matrix-Gruppen sind solche, die aus semi-direkten Produkten mit $\bF^n$ entstehbar. \\
Sei $G \subset\eMo\of{V}$ \kla{$V \cong \bF^n$} eine Gruppe \& $\fDef{A}{\bR\supset \of{-\epsilon,\epsilon}}{G}$ differenzierbare Kurve mit $A\of{0} = \id$. Wir definieren $\dot{A} \mDef \evLi{\dd{s}}_{s=0} A\of{s} =$ Tangentialvektor an der Gruppenidentität.
\begin{satz}
	Die Menge der Tangentialvektoren an die Gruppenidentität bilden eine \hl{reelle} Lie-Algebra, $\lie\of{G}$.
\end{satz}
\begin{proof}
	\begin{enumerate}
		\item Linearität: Ist $X = \dot{A}$ und $Y = \dot{B}$, definiere $C\of{s} = A\of{s}B\of{s}$, dann $\dot{C} = \dot{A} + \dot{B} = X + Y$. Ebenso: Ist $X = \dot{A}$, definiere $B\of{s} = A\of{as}$ $\dot{B} = a X \fuerall a \in \bR$. Geschwindigkeit bei $e\in G$ bildet Vektorraum über $\bR$.
		\item Abgeschlossenheit unter Kommutatorbildung: Sei $X = \dot{A}$ \& $Y = \dot{B}$. Wir müssen zeigen, dass eine Kurve $C\of{s}$ existiert mit $C\of{0} = e$ und $\dot{C} = \eof{X,Y} = XY - YX$. Definiere also 
		\newcommand{\oftau}{\of{\tau\of{s}}}
		\begin{equation}
			C\of{s} = \begin{faelle} A\oftau B\oftau A^{-1}\oftau B^{-1}\oftau & s \geq 0 \\ B\oftau A\oftau B^{-1}\oftau A^{-1}\oftau & s\leq 0 \end{faelle}
		\end{equation}
		wobei $\tau\of{s} = \sign\of{s} \sqrt{s}$ und invers $s\of{\tau} = \sign\of{\tau}\tau^2$
		\begin{figure}[H]
			\centering
			\begin{tikzpicture}
				\draw[->] (-5,0) to (5,0);
				\draw[->] (0,-2.5) to (0,2.5);
				\node[draw=none] at (5.2,0) {$s$};
				\node[draw=none] at (0,2.7) {$\tau$};
				\draw[domain=0:1,smooth] (-5,-2.5) plot ({-\x*\x*5},{-\x*2.5}) plot ({\x*\x*5},{\x*2.5});
			\end{tikzpicture}
		\end{figure}
		Obwohl keine der Kurven $s \mapsto A\oftau$ \etc selber differenzierbar ist \kla{weil $\tau\of{s}$ nicht differenzierbar ist}, ist dennoch die Kurve $s\mapsto C\of{s}$ bei $s = 0$ differenzierbar. Für $s \searrow 0$ \kla{Rechtsableitung} gilt:
		\begin{equation}
			\begin{split}
				\dot{C}_R = \lim\limits_{s\searrow 0}\cof{\frac{C\of{s}-e}{s}} & = \lim\limits_{s\searrow 0} \cof{\frac{\eof{A\oftau,B\oftau}A^{-1}\oftau B^{-1}\oftau}{s}} \\
				& = \lim\limits_{\tau\searrow 0} \cof{\eof{\frac{A\of{\tau}-e}{\tau},\frac{B\of{\tau}-e}{\tau}}A^{-1}\of{\tau}B^{-1}\of{\tau}} \\
				& = \eof{X,Y}
			\end{split}
		\end{equation}
		$\dot{C}_L$ analog. 
	\end{enumerate}
\end{proof}
Da die Lie-Struktur durch die von $\eMo\of{V}$ induziert wird, gilt automatisch die Jacobi-Identität. \\
Ist $D \in \hom\of{G,\opGL\of{W}}$ eine lineare Darstellung von $G$ auf den Vektorraum $W$, dann induziert diese eindeutig eine Darstellung
\begin{equation}
	D_{*} \in \hom\of{\lie\of{G},\eMo\of{W}}
\end{equation}
Das sieht man so: Sei $A\of{s}$ Kurve in $G$ mit $A\of{0} = e$ und $\evLi{\dd{s}}_{s=0} A\of{s} = X$. Dann ist $A'\of{s} = \of{D\circ A}\of{s} = D\of{A\of{s}}$ eine Kurve in $\opGL\of{W}$ mit $A'\of{0} = \evLi{e}_{\opGL\of{W}}$. Wir sehen voraus, dass $D$ differenzierbar ist.